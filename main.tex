\documentclass[12pt]{book}
\usepackage[utf8]{inputenc}
\usepackage[margin=1in]{geometry}
\usepackage{amsmath}
\usepackage{graphicx}
\usepackage{setspace}
\usepackage{biblatex}
\usepackage{mathptmx}
\addbibresource{Mendeley.bib}

\renewcommand{\rmdefault}{ptm}

\title{NSF GRFP Personal Statement}
\author{Matthew Moreno}
\pagenumbering{gobble}

\usepackage{ragged2e}
\setlength{\RaggedRightParindent}{\parindent}


\setlength{\parskip}{0.25em}

\hyphenpenalty=10000
\exhyphenpenalty=10000

\begin{document}

%\section{Introduction/Self-Reflection}

My scientific interest straddles disciplinary boundaries between biology and computing and, more broadly, between engineering and science.
I gravitate to the intersection of exploration and creation.
I see a parallel intersection in music.
Science is like studying the theory and history of a piece.
Engineering is like rehearsing and performing it.
Exploring and creating inform each other and, together, are electric.
In sixth grade, when I began playing the oboe, I got my start in biology.
I remember staying up late to hot glue tiny beads from my family's junk drawer cardboard cell model I had traced out.
All the while, I struggled to piece together WHAT from the rather simplistic descriptions of organelles we had learned.
My initial reaction was disbelief.
In the next breath, though, I began thinking down a different track.
Evidence that systems governed by random interactions between innumerable independent components can support cohesive and dynamic functionality permeates our living world.
It hit me: answers to the question ``how?'' do too.
From that point onward, the astonishing structures and mechanisms embedded at every level of biological organization have continuously delighted me.
I have found studying biology to be exploration in the truest, most adventurous, sense of the word.
My career aspiration is in the field bio-AI, harnessing biologically-inspired approaches to create intelligent systems.
I believe that my unique set of interests, straddling between biology and computing and, more broadly, between engineering and science, set me apart and will continue to distinguish my work.


\noindent
\underline{\smash{\textit{\textbf{Research History}}}}
I began gaining research experience as a high school student through an \textbf{Apprenticeships in Science program} at Oregon State University.
I worked with Dr. John Folwer to investigate the exocyst complex in plants with the Fowler lab at Oregon State University.
Used genetic assays and phenotypic measurements, I screened 25 populations of \textit{Arabadopsis thaliana} for synergistic interaction between a known exocyst mutation and a mildly deleterious mutation at an independently-assorting locus.
The group went on to isolate a novel Golgi-localized protein confirmed to interact with the exocyst \cite{fowler}.
I took away a strong impression of what a friendly, supportive, and inclusive scientific community looks like.
This personal experience with STEM outreach motivates my desire to make similar opportunities available to others.

Early in my college career, I worked two full-time summers with the \textbf{USDA small fruits breeding laboratory}.
The group, led by Dr. Chad Finn, develops berry cultivars for the fresh fruit and processed food markets.
Realizing widespread production of those cultivars requires buy-in from our partner growers.
To build these relationships, we hosted regular symposiums with brief and actionable presentations on best practices and actively involved growers in our scientific experiments.
Later in my undergraduate career, I exercised my experience at the interface of science and industry through three bouts in the \textbf{Mathematical Competition in Modeling} (MCM).
These four day sprints emphasize pitching insights from mathematical models participants develop and analyze to business executives and policy makers.
In 2017, my three-person team developed a model of traffic in the greater Seattle area and showed that in the near future designating lanes exclusively for autonomous vehicles will reduce commuter travel delays.
We \textbf{ranked among the top 0.8\% of over 1,500 participating teams}.
These experiences taught me how to gear science towards a professional audience and build collaborative partnerships.
I will use these skills to translate my research into real-world innovations.

As a sophomore, I worked with Dr. Adam Smith at University of Puget Sound (UPS) to \textbf{develop methods for automated isolation of mouse
ultrasonic vocalizations (USVs) from noisy recordings}.
I sought outside funding and was \textbf{awarded a NASA space grant of \$3,250} through a competitive application process.
USVs are an important quantitative assay for the affective and social state of mice in biomedical research and existing software tools were readily confounded by background noise.
I developed and tested filtering algorithms inspired by the Sobel Edge detection method that use human-annotated spectrograms to learn to distinguish between true mouse vocalization signals and background noise.
My approach achieved 75\% accuracy at 25\% recall from noisy recordings.
I presented these results at the UPS summer research symposium \cite{smith}.
I took away concrete computational research skills including data management, version control, and visualization techniques.

In the summer after my junior year, I brought together my biological and computational interests together studying ant foraging behavior at the \textbf{NJIT Swarm lab}.
I was recruited by Drs. Simon Garnier and Jason Graham through a Mathematical Biosciences Institute REU.
On flat terrain, the shortest-distance foraging path, the most energy-efficient path, and the quickest path are all identical.
However, on uneven terrains an obstacle may make the most direct path take longer than a trip that circumvents it.
In the absence of an absolute ``best'' path, the question of how ants make trade-offs is of great interest.
Thus, I extended computational models of ant foraging to consider uneven terrains.
My differential-equations based model predicts that severe inclines cause ants to favor a more direct, less variable foraging path.
I presented my work at the \textbf{2017 Joint Mathematics Meetings} \cite{jmm}.
During my time at the Swarm lab, I found the autonomy entrusted to me and the opportunity to answer open biological questions with practical applications like swarm robotics empowering and rewarding.
After this REU, I saw graduate school as the best way to continue engaging in such self-directed, important work.

For my \textbf{senior thesis project} at UPS, I worked with Dr. Chambers to synthesize a conceptual framework to understand evolvability, the potential for adaptive change to occur in an organism's descendants.
Evolvability has been connected to a large array of causal factors.
I organized this constellation of causal factors into three larger classes and analyzed their broad relationships \cite{thesis}.
Building off my thesis, in my subsequent \textbf{capstone project} I used a gene-regulatory network model to empirically investigate the how environmental influence on the phenotype and relates to evolvability.
I found that populations subjected to stochastic perturbations of the development process evolved a higher incidence of silent mutation and that populations evolved to respond to environmental signals by activating alternate developmental pathways were more sensitive to mutation.
I presented my capstone and thesis work at the 2016 NW Honors Symposium in Seattle, two campus seminars, and the 2017 BEACON Congress \cite{beacon}.
For my thesis and capstone work, I received the \textbf{MacArthur Award for an Outstanding Thesis Presentation} and the \textbf{Goman Outstanding Math/CS Senior Award}.
This fall, I have published my thesis work as an illustrated blog series aimed at both the general public and other scientists.
I found uniting experiments and theory in my thesis and capstone projects extremely rewarding;
this work inspired me to continue conducting research that tightly couples these domains.


\noindent
\underline{\smash{\textit{\textbf{Broader Impacts}}}}
I developed educational outreach skills as \textbf{a subject tutor, an academic consultant, an AP tutor, a classroom assistant, a musical coach, and an after-school club leader}.
In these capacities, I worked with high school and college students on a weekly basis over three and a half years.
I see these mentoring relationships among the most important accomplishments of my undergraduate career.
I was caught off guard by an alternative high school student I tutored in math who, out of the blue, asked, ``What's so great about college?''
I almost launched into why \textit{I} liked college, but instead asked, ``What's your favorite thing to talk about?''
She took a moment, then replied, ``Books, I guess.''
``What's so cool,'' I told her, ``is that you can take classes in whatever you want.
You can take classes about literature with other people who love books too.''
She took notice when I mentioned how well she would get along with my humanities friends.
This experience made me realize the impact of making a personal connection and showing a student they belong.
I went on to \textbf{lead a CS mentoring initiative} in my senior year.
We organized three department-funded meet-ups to develop social connections between class cohorts and build skills like professional networking and programming in IDEs.
Each was attended by approximately fifteen students.

By showing me the difference \textit{I} can make, my outreach work
led me to define making higher education and STEM more welcoming and accessible as a key priority of my scientific career.
As an LGBT person, I know that identity is not baggage to be checked at the door.
I believe recognizing and respecting diversity makes STEM stronger.
I strive to use my own experiences to reach across barriers to inclusivity experienced by others in the scientific community.


\textbf{Career Goals}
My current research interests directly stem from my undergraduate thesis and capstone work.
I am pursing a doctoral degree with advisor Dr. Charles Ofria in the NSF BEACON Center for the study of evolution in action at Michigan State University.
The BEACON Center's high performance computing resources and interdisciplinary community uniquely position me collaborate to collaborate with biologists on \textit{in vivo} work and with evolutionary computing practitioners to apply novel biological concepts to evolutionary computation.
I am interested in understanding what critical nuances are missing from simple algorithmic implementations of evolution built on bare bones selection, variation, and inheritance;
my specific focus is on environmental influence in the genotype-phenotype mapping.

By probing the algorithmic principles of biology I hope to contribute to the development of more capable and versatile AI systems.
When reflecting on the potential human impact of advances in computing, I am reminded of my father, who volunteers with the Dial-A-Bus program in Benton county.
He serves individuals who would be otherwise unable to get around town, curtailing their ability to fully participate in the community.
I hope that, through applications that counteract disability and free us from dangerous or simply menial tasks, stronger AI will enable people to more fully exercise their human capabilities.
I see glimmers of progress towards this goal in the work I've done with the Swarm Lab, Dr. Chambers, and, now, the Ofria lab.
I will use the skills I gained through the USDA and the MCM to apply my research to real-world AI problems.

I am committed to continuing my work fostering community among scientists and performing STEM outreach.
I currently volunteer four hours a week as a teacher's assistant in special and general education classrooms in East Lansing.
As I progress in my graduate studies, I look forward to welcoming new graduate students and taking advantage of BEACON's funding support to mentor undergraduates on summer research projects.
I am concerned about barriers to inclusivity in the scientific community.
My experience as a LGBT person has deepened my perspective on these barriers.
I believe identity is not baggage to be checked at the door but must be recognized and respected.
Although the barriers I have encountered pale in comparison to those faced by others, I hope to leverage these molehills to reach out over mountains.


\textbf{Conclusion}
My affiliation with BEACON provides a strong foundation for the goals I have laid out for my graduate career.
My field --- particularly the BEACON community --- is rife with opportunities for collaborations across disciplines and institutions.
I am particularly interested in collaborating with biologists to perform wet work to better understand what properties of the genotype-phenotype mapping facilitate evolution and working with to hardware/electrical engineers to explore distributed system design for swarm computing.
GRFP funding would provide a strong position to develop and take advantage of relationships with science and industry collaborators.
In graduate school, I want to facilitate similar experiences for other students through near-peer mentoring relationships.
BEACON provides opportunities, particularly summer research funding for dozens of undergraduates from underrepresented backgrounds, to develop these relationships.
I will invest time and resources made available by GRFP support into these relationships.
I feel strongly that education, outreach, mentorship, and community are intrinsic to what it means do science.
At a scale larger than myself, I believe the scientific community stands to benefit by affording the same intrinsic value we place in the traditional, intellectual aspects of science to these activities and even more deeply incorporating them into our work.
GRFP support help me lead in the conversation about the intrinsic value of outreach, education, mentorship, and community in science.
I see my graduate career as an unparalleled opportunity to --- harnessing a more nuanced understanding of evolution to a more fleshed-out scientific appreciation for evolution in nature and practical problem-solving computational techniques and strengthening the intellectual community doing this work.
The latitude GRFP support provides, in conjunction with dedicated resources available through the NSF BEACON, will kick-start my graduate ambitions.


\end{document}


%Ant colonies regulate their foraging behavior through a collective decision making process; when ants forage, no individual ant operates with complete information about the terrain they are exploring.
% This contrasts with traditional human approaches decision-making, which typically centralize information, process it, then redistribute instructions.
% Consider, for example, traffic-aware navigation tools such as Waze or Google Maps; the distribution of traffic distribution across a geographic region is collected from users, centrally processed, and then routing instructions are redistributed to individual users.
% In contrast, collective intelligence on the level of the ant colony emerges from parallel execution of a simple set of individual pheromone deposit and response behaviors; among other feats, foraging ants will tend to to choose the shortest path between nest and food and to selectively exploit the richest of an array of food sources.
