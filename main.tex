\documentclass[12pt]{book}
\usepackage[utf8]{inputenc}
\usepackage[margin=1in]{geometry}
\usepackage{amsmath}
\usepackage{graphicx}
\usepackage{setspace}
\usepackage{biblatex}
\usepackage{mathptmx}
\addbibresource{Mendeley.bib}

\renewcommand{\rmdefault}{ptm}

\title{NSF GRFP Personal Statement}
\author{Matthew Moreno}
\pagenumbering{gobble}

\begin{document}

In sixth grade, I was up late to glue tiny beads from my family's junk drawer onto the cardboard cell model that I had traced out.
While I struggled to place the small plastic pieces onto my model, I also struggled to link the simplistic descriptions of organelles we had learned into an understanding of the cellular whole.
Because every piece of the cell depends on others, I felt I could not truly understand any piece in isolation.
That experience stoked a glimmer of curiosity that grew into a love of thinking about and experimenting with abstractions of the \textbf{astonishing complex structures and mechanisms embedded at every level of biological organization}.
Today, \textbf{computational experiments} are my vehicle for this endeavor.

When I started to develop an interest in biology, I also began playing the oboe.
Like science, band culture is built on \textbf{mentoring relationships, cooperative competition, and friendly warmth between peers}.
Without this community, we could not develop extensive technical skills, push ourselves to do our best work, and accomplish goals far beyond the capabilities of any single individual.
The willingness to invest time and energy, high expectations, and mutual respect scientists hold for their kind inspired me to pursue a career in science.

As a scientist, I aspire
(1) to \textbf{develop fundamental theory and applications for bio-inspired approaches to AI} and
(2) to \textbf{promote an inclusive and intellectually vibrant science community through mentoring relationships and STEM education}.


%\section{Research History}
My first foray into research was with the Fowler Laboratory at Oregon State University during the summers of 2011 and 2012.
I recall, while examining the translucent roots of a few dozen delicate seedlings on parafilm-sealed petri dishes for characteristically stumpy mutants, experiencing a sense of exhilarating self-awareness; I relished the purpose, discipline, and curiosity of my work.
My research goal was to identify proteins that interacts interact with the exocyst complex.
I screened populations of \textit{Arabidopsis thaliana}, a model organism in plant biology, for a 13:3:1 phenotypic ratio telltale of synergistic interaction between a known mutation to the exocyst complex and a mildly deleterious mutation at an independently-assorting locus.
After performing a statistical analysis of my observations, I flagged two populations for further analysis.
My mentor, research assistant Rex Cole, and I hoped that identification of novel exocyst interactors, or confirmation of known exocyst interactors, would shed light on the role of the exocyst in plants.
Although not immediately translating to technological or commercial ends, as fundamental research such work is key to opening avenues of inquiry for future study and, ultimately, in paving the way for other more applied advances.
After my time at the Fowler lab, I continued working in botany and horticulture for two years, studying the molecular biology of vineyard grape development with the Deluc laboratory and developing berry cultivars for the fresh fruit and processed food markets with the USDA Finn group.
From my work in laboratory biology, I became versed in the carefully-considered design and methodological meticulousness experimental inquiry requires, which, I believe, gives me an edge in the domains of computer science and mathematics and which I look forward to continuing to put to use in graduate school.

Fast forward to my REU with the Swarm Lab at NJIT last summer.
I sat at my new desk in the open-plan office after a preliminary meeting with visiting mathematics professor Jason Graham and primary investigator Simon Garnier consumed by a swirl of surprise, adrenaline, and, also, a little panic.
The meeting boiled down to this:
``Here's the problem.
Here are some resources.
Take a crack at it.''
Although I was initially surprised and intimidated by the extent of the autonomy and discretion they entrusted to me, in retrospect I wouldn't have had it any other way; owning my work was personally rewarding and, I believe, ultimately contributed to its quality.
My task was to extend mathematical models of ant foraging, which are well-developed on flat surfaces, to uneven terrains.
It is well known that, as a collective, ants can optimize the foraging path they travel between nest and food.
On flat terrain, a clear ``best'' path exists: the shortest-distance foraging path, the most energy-efficient path, and the quickest-trip path are all identical.
On uneven terrains, however, this is no longer necessarily the case and the question of which trade-offs ants make --- and how they make them --- is of great interest.
After surveying existing models of ant foraging behavior on flat terrain and individual ant behavior on inclined surfaces, I designed and numerically evaluated a differential equations-based model of the foraging behavior of ants over uneven terrain.
Analysis of time series of ant positions and orientations generated via simulation revealed that as the severity of the incline the ants traverse increases, the model predicts that ants will tend to favor a more direct and less variable foraging path.

My model will see continued use, enabling the Swarm Lab to work out the rules by which real ants act on uneven terrain in a foraging context by simulating hypothesized behaviors and assessing the resulting foraging path predictions in comparison with upcoming in vivo ant foraging experiments.
Ultimately, research into the collective intelligence of insects translates directly to technological applications; for example, such research has been leveraged in swarm robotics projects, such as NASA's ant-inspired ``Swarmies'' that may one day harvest resources for Martian colonies and distributed traffic management schemes (especially in relation to autonomous vehicles).
On a personal level, this research experience has translated into a penchant for autonomously chewing on difficult problems, which will serve me well throughout graduate school.


%\section{Research Interests}
%
% I stumbled into my current research interests while waiting at an interminable red light on my bike four years ago.
% I reasoned that the dynamics of traffic flow through a network of irregular intersections are so complicated and nuanced that an optimal (or, at least, near-optimal) system lies beyond the reach of traditional heuristic design approaches.
% An article I had encountered years prior surfaced in my memory.
% It documented efforts to generate field-programmable gate array configurations well-adapted to certain tasks through repeated selection, recombination, and mutation of candidate configurations.
% This type of work is part of the larger field of evolutionary computation, which aims to employ algorithms inspired by evolution for problem-solving ends.
%
% I toyed with the problem for a few days --- developing an approach that, in retrospect, falls under the purview of linear genetic programming --- and then, after working up the courage, booted up my text editor and jumped headfirst into implementing it.
% Although my interest in evolutionary computing began as a hobby, in short order I fell down the rabbit hole.
% It was exhilarating to see that the ideas I was exploring were of interest to others.
% Inspired by scholarly literature on genetic programming, I switched my evolutionary algorithm to tournament-based selection and incorporated page-based recombination.
%
% Although traffic control had originally sparked my interest, I quickly switched to working in other, more standard, problem domains such as pole balancing and maze exploration.
% Receiving guidance from my undergraduate advisor Professor Adam Smith, with whom I had developed tools to extract mouse vocalizations from noisy recordings during the summer of 2015, I wrote a mock project proposal laying out the design for a system of concurrent instruction readers to parse a linear instruction set.
% The aim of this scheme was to secure a genotype-phenotype encoding more robust to mutation and recombination.
% Carrying out this project was an invaluable opportunity to learn and explore by playing in the sandbox.

My current interest is in using computational models to better understand evolution.
In this realm, understanding evolvability  ---  promoting the generation of useful variation during the evolutionary process  ---  is a difficult, but important, task.
Investigating evolvability amounts to digging into the relationship between the configuration of a system and the outcomes of mutational perturbation to that system.
My senior thesis, advised by Professor America Chambers, focused on synthesizing a conceptual framework for evolvability.
My subsequent capstone project investigated the relationship between environmental influence on the phenotype and evolvability.
In experiments performed with a gene regulatory network model, I found that populations evolved under a regime stochastic environmental perturbation of the developmental process were more resistant to mutation.
Using the same model, I also found that populations evolved under a selective pressure to respond to environmental signals by activating alternate developmental pathways were more sensitive to mutation.
I hypothesize that, in these cases, environmental influence on the phenotype led to selection for certain internal structural characteristics that modulate the phenotypic consequences of that environmental influence.
These internal structural characteristics, in turn, affect mutational outcomes.
Probing the question of evolvability has immediate implications to evolutionary computing, potentially yielding more powerful digital evolution techniques, in addition to relevance to the field of evolutionary biology, which is grappling with the relationship between environmental influence on the phenotype and evolution.


%\section{Career Aspirations}
Although my interest lies in fundamental research, by probing the algorithmic principles of biology I hope to contribute to their harnessing for the development of more capable and versatile artificial intelligence systems.
Beneath my own curiosity and ambition, I feel a moral imperative to participate in this work.

When reflecting on the human impact of artificial intelligence, I am reminded of my father, who volunteers with the Dial-A-Bus program in Benton county.
He serves individuals who would be otherwise unable to get around town, curtailing their ability to fully participate in the community.
I hope that, through applications that counteract disability and free us from dangerous or simply menial tasks, stronger artificial intelligence will enable people to more fully exercise their human capabilities and, thereby, lay a fuller claim to their humanity.

In my experience, the motivations of scientists --- especially those working in technology-driven fields --- can often come off as cerebral and abstract.
(I recall, in particular, attending a talk that closed with a back-of-a-napkin calculation of the limit imposed by relativistic constraints on the maximum number of human lives we could create over the lifespan of the universe).
I strive to check this tendency in myself by engaging directly with causes I believe in, such as youth outreach.
Over the course of my undergraduate career, I have jumped at many opportunities to get to know and mentor Tacoma-area students: as an AP tutor, a classroom assistant, a musical coach, a Q\&A panelist, and an after-school club leader.
I feel very strongly about affording other students the same opportunities to exercise their capabilities in music, science, and math that I have enjoyed.


Ultimately, I see continuing in academic research to lead a research group as the best route to build the type of community I have enjoyed belonging to and accomplish my outreach, technological, and scientific goals.
GRFP support will help me not only conduct cutting edge research at the intersection of biology and computer science, but also to lead in the conversation about the intrinsic value of outreach, education, mentorship, and community in science.


\end{document}


%Ant colonies regulate their foraging behavior through a collective decision making process; when ants forage, no individual ant operates with complete information about the terrain they are exploring.
% This contrasts with traditional human approaches decision-making, which typically centralize information, process it, then redistribute instructions.
% Consider, for example, traffic-aware navigation tools such as Waze or Google Maps; the distribution of traffic distribution across a geographic region is collected from users, centrally processed, and then routing instructions are redistributed to individual users.
% In contrast, collective intelligence on the level of the ant colony emerges from parallel execution of a simple set of individual pheromone deposit and response behaviors; among other feats, foraging ants will tend to to choose the shortest path between nest and food and to selectively exploit the richest of an array of food sources.
