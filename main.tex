\documentclass[12pt]{book}
\usepackage[utf8]{inputenc}
\usepackage[margin=1in]{geometry}
\usepackage{amsmath}
\usepackage{graphicx}
\usepackage{setspace}
\usepackage{biblatex}
\usepackage{mathptmx}
\addbibresource{Mendeley.bib}

\renewcommand{\rmdefault}{ptm}

\title{NSF GRFP Personal Statement}
\author{Matthew Moreno}
\pagenumbering{gobble}

\begin{document}

Smoke might as well have been coming out of my ears.
When I'm thinking, really thinking, long pauses between clauses throw a jagged edge into my cadence.
I'll tap my forehead, squint up and turn towards the ceiling, even work through a sentence in several drafts before announcing that, yes, that's what I want to say.
I had come in to meet with my undergraduate thesis advisor, Professor America Chambers, ready to discuss several papers on the Hyper-NEAT encoding scheme and derivative techniques.
We were up at the whiteboard in her office, walking through an illustrative example employing the Hyper-NEAT scheme.
She sat down, put her chin in her hand, and piercingly scrutinized the board.
``So, why would you want to do this? What does it get you?" she asked.
``Well,'' I started.
``Well.
Hmmm.
It's reorganizing... rearranging the genotype-phenotype mapping.''
I turned to the board and rolled the Expo marker in my palm.
Then, speaking more quickly,
``What it really does... it biases the search space.
It makes regular phenotypic outcomes more easily accessible, more likely to occur...
So for problems where good solutions exhibit phenotypic regularity, those solutions become easier to find.''
We continued back and forth, keeping on for a few more minutes.

As I was packing up, Dr. Chambers grinned and asked ``When you're with your friends, do you think so hard about your words?''
I took a few seconds to consider.
``No, I don't think so.
Not as much, anyways.''
It wrings me out, but this is what I love.
I love topics that trip me up and conversations that make me grope for my words.
I especially love plugging into the challenging, collaborative intellectual environment cultivated in the scientific community to distill diffuse curiosity into direct, actionable questions.

My current research interest is seeking out what's missing from simple descriptions of evolution built on bare bones selection, variation, and inheritance.
Working with advisor Dr. America Chambers, I synthesized a conceptual framework to understand evolvability as my undergraduate thesis work.
Evolvability, the relationship between the configuration of a system and the phenotypic outcomes of mutational perturbation to that system, has been connected to an overwhelmingly large array of causal factors.
I organized this constellation of causal factors into three larger classes and described broader relationships between these classes.
My subsequent capstone project empirically investigated the relationship between environmental influence on the phenotype and evolvability.
In experiments performed with a gene regulatory network model, I found that populations evolved with stochastic perturbations to the development process had a higher incidence of silent mutation.
I hypothesize that modeling disruptive environmental influence on the phenotype introduced selection on internal structural characteristics --- specifically, against unstable cyclic regulatory interactions --- that otherwise would have remained invisible.
These internal structural characteristics, in turn, affect the phenotypic outcomes observed under mutation, in this case increasing the frequency at which mutations had no phenotypic effect.
Working with the same model, I also found that populations evolved under a selective pressure to respond to environmental signals by activating alternate developmental pathways were more sensitive to mutation.

My capstone and thesis work contribute to active conversations about evolvability taking place among and between evolutionary computing and evolutionary biology researchers.
Evolvability has immediate practical implications to evolutionary computing.
Classic techniques, such as genetic programming, are largely built on the bare bones understanding of evolution.
Translating more nuanced elements of the evolutionary process like environmental influence on the phenotype, by increasing the novelty and usefulness of mutational outcomes, will yield more powerful digital evolution techniques.
Evolutionary biology, in turn, is grappling with a number of recent, controversial extensions to the canonical modern synthesis.
Key to evaluating these extensions are questions of the evolutionary origins of evolvability and the role of environmental influence on the phenotype in the evolution of novel traits.
Computational techniques offer a unique ability to explicitly state assumptions and test if they have sufficient explanatory power.

My first run-in with the exciting power of computational techniques to address biological questions was with the Swarm lab at NJIT.
I sat at my new desk in the open-plan office after a preliminary meeting with visiting mathematics professor Jason Graham and primary investigator Simon Garnier consumed by a swirl of surprise, adrenaline, and, also, a little panic.
The meeting had boiled down to this:
``Here's the problem.
Here are some resources.
Take a crack at it.''
Although I was initially surprised and intimidated by the extent of the autonomy and discretion they entrusted to me, in retrospect I wouldn't have had it any other way;
ownership of my work was personally rewarding and, I believe, ultimately contributed to its quality.
My task was to extend mathematical models of ant foraging, which are well-developed on flat surfaces, to uneven terrains.
It is well known that, as a collective, ants can optimize the foraging path they travel between nest and food.
On flat terrain, a clear ``best'' path exists: the shortest-distance foraging path, the most energy-efficient path, and the quickest-trip path are all identical.
On uneven terrains, however, this is no longer necessarily the case.
For example, an obstacle may make a trip along the most direct path take longer than a trip that circumvents it.
In the absence of a singular ``best'' path, the question of which trade-offs ants make --- and how they make them --- is of great interest.
After surveying existing models of ant foraging behavior on flat terrain and individual ant behavior on inclined surfaces, I designed and numerically evaluated a differential equations-based model of the foraging behavior of ants over uneven terrain.
Analysis of time series of ant positions and orientations generated via simulation revealed that as the severity of the incline the ants traverse increases, the model predicts that ants will tend to favor a more direct and less variable foraging path.

My model will see continued use, enabling the Swarm Lab to understand how real ants make collective foraging decisions on uneven terrain by simulating hypothesized behaviors and assessing the resulting foraging path predictions in comparison with upcoming \textit{in vivo} foraging experiments.
Ultimately, research into the collective intelligence of insects translates directly to technological applications;
for example, such research has been leveraged in swarm robotics projects, such as NASA's ant-inspired ``Swarmies'' that may one day harvest resources for Martian colonies and distributed traffic management schemes (especially in relation to autonomous vehicles).
On a personal level, I strengthened my self-confidence and self-management.
Importantly, I also learned when and how to ask for help.
I felt empowered by the process of translating fragmentary ideas into a vision and then into results.
During my graduate career, I am looking forward to continuing to learn to operate this idea-vision-results pipeline and, especially, having the free reign to run it full bore.

Science community stuff here.
In sixth grade, when I got my start in biology, I also began playing the oboe.
Band kids root in an intense culture built on cooperative competition, mentoring relationships, and friendly warmth between peers.
Several years later, when I joined up with a botany research group, I felt right at home.
These values are perpetuated among scientists and musicians alike by tradition and necessity.
How else can we develop extensive technical skills, push ourselves to do our best work, and come together to build something larger than ourselves?
I love the willingness to invest, high expectations, and mutual respect scientists and musicians alike hold for their kind.
It was these values that inspired me to turn my interest in science into a career in science.
Independent of passion for my research interests, I cherish my membership in the scientific community in and of itself.
To me, belong as a scientist acting is completely entwined with acting to further these values and to bring new members into our fold.

Talk more about inclusion/social here (graduate student who clearly took the time to make high school students working part time feel part of the group, Rex Cole, etc.)
My research goal was to identify proteins that interact with the exocyst complex.
We worked with populations of \textit{Arabidopsis thaliana}, a model organism in plant biology, that harbored a known mild mutation directly affecting the exocyst complex.
I screened these populations for a 13:3:1 phenotypic ratio, which would be telltale of synergistic interaction between the known exocyst mutation and a mildly deleterious mutation at an independently-assorting locus.
This entailed examining the translucent roots of dozens of delicate seedlings on a parafilm-sealed petri dishes for characteristically stumpy mutants.
I relished the purpose and discipline of this work.
After performing a statistical analysis of my observations, I flagged two populations for further analysis.
By identifying new exocyst interactors, we aimed to shed light on the role of the exocyst in cellular biology.
Although not immediately translating to technological or commercial ends, as fundamental research such work is key to opening avenues of inquiry for future study and, ultimately, in paving the way for other more applied advances.
After my time at the Fowler lab, I continued working in botany and horticulture for two years, studying the molecular biology of vineyard grape development with the Deluc laboratory and developing berry cultivars for the fresh fruit and processed food markets with the USDA Finn group.
From my work in laboratory biology, I gained expertise in experimental design and methodology.
I have found that this expertise sets me apart in the domains of computer science and mathematics.
I aim to leverage this edge throughout my graduate research career.

Over the course of my undergraduate career, I built relationships with younger students at the University of Puget Sound and the Tacoma School District as a subject tutor, an academic consultant, an AP tutor, a classroom assistant, a musical coach, a Q\&A panelist, and an after-school club leader.
For me, these relationships became part of what it meant to be a student.
Today, I see them as the most important accomplishments of my undergraduate career.

In particular, I remember being caught off guard by a student I was talking to about triangles at Oakland, an alternative high school.
Out of the blue, she asked, ``What's so great about college?''
In that moment, I almost launched into a biographical account of why \textit{I} liked it.
Instead, I asked, ``What's your favorite thing to talk about?''
She took a moment, then replied, ''Books, I guess.''
She elaborated on her interest for stories and characters, elaborating on a few of her favorite pieces of young adult fiction, and bragged a little about reaching the cap for books that could be checked out of the library at once.
``What's so cool,'' I told her, ``is that you can take classes in whatever you want.
You can take classes about stories and hang out with other people who love them too.''
We went back and forth about \textit{really} being able to study what you want a few times before she seemed somewhat satisfied.
Before returning to triangles, I suggested that she would really get along with my humanities friends studying literature,
``That's really what they like to talk about.
All the time.
You'd fit in.''
She helped me realize that sharing a personal connection, a feeling, or enthusiasm is often more important than dispensing advice and information.
I began to recognize similar dynamics with the STEM students I tutored.
At the end of an some of my appointments, our conversation would shift to upper division coursework, techniques, and concepts that lay years ahead for them.
I took these conversations as opportunities to establish us as members of the same intellectual community instead of opportunities to dispense graduation requirements and course descriptions.
Although these were small gestures, I feel proud to have to put out the metaphorical welcome mat.

In graduate school, I want to facilitate similar experiences for other students through near-peer mentoring relationships.
The NSF BEACON Center provides opportunities, particularly summer research funding for dozens of undergraduates from underrepresented backgrounds, to develop these relationships.
I feel strongly that education, outreach, and mentorship are intrinsic to what it means do science.
At a scale larger than myself, I believe the scientific community stands to benefit by affording the same intrinsic value we place in the traditional, intellectual aspects of science to these activities and even more deeply incorporating them into our work.

For example, I began drawing freehand cartoons illustrating my work for presentations and articles aimed at a broader audience.
To my surprise, I found they became important tools to engage a scientific audience, as well.
They have even become useful to my own understanding of topics I'm working on.


Although intellectual impact can be made tangible through technology, as scientists we must also recognize the immediate impacts we can make through choices about how we choose to conduct ourselves and our research activities.

Although my interest lies in fundamental research, by probing the algorithmic principles of biology I hope to enable the development of more capable and versatile artificial intelligence systems.
Beneath my own curiosity and ambition, I feel a moral imperative to participate in this work.
Motivations of scientists --- especially those working in technology-driven fields --- can often come off as cerebral and abstract.
(I recall, in particular, a talk that closed with a back-of-a-napkin calculation of relativistic constraints on the maximum number of human lives that could be created over lifespan of the universe).
I strive to check this tendency in myself.

When reflecting on the potential human impact of advances in computing, I am reminded of my father, who volunteers with the Dial-A-Bus program in Benton county.
He serves individuals who would be otherwise unable to get around town, curtailing their ability to fully participate in the community.
I hope that, through applications that counteract disability and free us from dangerous or simply menial tasks, stronger artificial intelligence will enable people to more fully exercise their human capabilities and, thereby, lay a fuller claim to their humanity.
I see glimmers of progress towards this goal in the work I've done with the Swarm Lab, the Fowler lab, and, now, the Ofria lab.

To translate my research into applied advances in artificial intelligence, I will work in close collaboration with industry.
I learned what working with industry means during my my time with the Finn small fruits breeding laboratory at the USDA.
It means making long trips to their homestead to exchange biological samples and chitchat.
It means walking growers through your field and hawking animated accounts of the up and coming crosses they should keep an eye out for.
It means organizing regular symposiums around brief and actionable presentations on best practices.
Of course, it means making a point to wrap up those symposiums by taste-testing berry sauce with ice cream.
It means continually brainstorming catchy names for new cultivars that memorably allude to their best traits.
It means tabulating thorn density on 10cm lengths of cane in service of patent applications for those cultivars
I know how to develop personal relationships then concretely explain --- and demonstrate --- a compelling use case for my work in the context of \textit{their practices} then make it easy --- and exciting --- to incorporate my work into their practices.
Collaborating with industry also means acres and acres of new berry cultivars plantings that over the horizon; it means new berry cultivars in muffins, ice cream, farmers markets, and roadside stands.
I am eager to continue to exploit collaboration with industry during my graduate career and beyond.


I also care about taking ownership of my idea-vision-results pipeline.
My field --- particularly the BEACON community --- is rife with opportunities for collaborations across disciplines and institutions.
I am particularly interested in collaborating with biologists to perform wet work to better understand what properties of the genotype-phenotype mapping facilitate evolution and working with to hardware/electrical engineers to explore distributed system design for swarm computing.
GRFP funding would provide a strong position to develop and exploit these relationships.
As I begin my graduate career, I am grateful for the opportunity to join in exploring the intersection between computing and evolution and join up with the community doing this work.
GRFP funding will support my ambitions for my graduate career: to lead in strengthening the scientific community and to lead in organizing intellectual collaboration.

I am particularly inspired by the leadership of the NSF BEACON center in this direction by, for example, repackaging research software as educational tools and leading outdoor field trips for schoolkids to collect data.
GRFP support will help me lead by example to grow the conversation about what it means to do science.


The latitude GRFP support provides, in conjunction with dedicated resources available through the NSF BEACON center for the study of evolution in action , will kick-start my efforts.

I will invest time and resources made available by GRFP support into these relationships.
Be a leader in conversation about the intrinsic value of outreach and education and mentorship.


\textbf{Research Interests}
I am interested in understanding what critical nuances are missing from simple algorithmic implementations of evolution built on bare bones selection, variation, and inheritance.
As an undergraduate advised by Dr. America Chambers, I synthesized a conceptual framework to understand evolvability, the potential for adaptive change to occur in an organism's descendants.
The origins of evolvability have been connected to a large array of causal factors.
I organized this constellation of causal factors into three larger classes and analyzed their broad relationships.
For my subsequent capstone project, I used a gene-regulatory network model to empirically investigate the relationship between environmental influence on the phenotype and evolvability.
I found that populations subjected to stochastic perturbations to the development process evolved a higher incidence of silent mutation.
I hypothesized that disruptive environmental influence on the phenotype selected against unstable cyclic regulatory interactions that otherwise would have remained invisible, thus increasing the frequency at which mutations had no phenotypic effect.
Working with the same model, I also found that organisms evolved to respond to environmental signals by activating alternate developmental pathways were more sensitive to mutation.

My research contributes to ongoing scientific conversations about evolvability.
Evolutionary biologists see evolvability as central to understanding the evolution of complex traits such as human intelligence.
Questions about the role of environmental influence on the phenotype in the evolution of novel traits, in particular, are central to debate over recent, controversial extensions to the canonical modern evolutionary synthesis.
In turn, evolutionary computing researchers seek more evolvable systems to solve practical problems.
Translating more nuanced elements of the evolutionary process like environmental influence on the phenotype will yield powerful applied evolution techniques.
I am a doctoral student at Michigan State University, working with Dr. Charles Ofria in the NSF BEACON Center for the study of evolution in action.
I am uniquely positioned to contribute to conversations about evolutionary dynamics to apply novel biological concepts to evolutionary computation.


\textit{\textbf{Research History}}
I began gaining research experience in high school.
Through a local \textbf{Apprenticeships in Science program}, I investigated the exocyst complex in plants with the Fowler lab at Oregon State University.
I used genetic assays and phenotypic measurements to screen 25 populations of \textit{Arabadopsis thaliana} for synergistic interaction between a known exocyst mutation and a mildly deleterious mutation at an independently-assorting locus.
The group went on to isolate a novel Golgi-localized protein confirmed to interact with the exocyst \cite{fowler}.
I took away a strong impression of what a friendly, supportive, and inclusive scientific community looks like.
This personal experience with STEM outreach motivates my desire to make similar opportunities available to others.

Beginning my college career, I worked two full-time summers with the \textbf{USDA small fruits breeding laboratory}.
The group, led by Dr. Chad Finn, develops berry cultivars for the fresh fruit and processed food markets.
Realizing widespread production of those cultivars requires deliberate cooperation with partner growers.
Among other efforts, we hosted regular symposiums around brief and actionable presentations on best practices and made a point to always wrap up with taste-tests of berry sauce with ice cream.
Later in my undergraduate career, I exercised my experience at the interface of science and industry through three bouts in the \textbf{Mathematical Competition in Modeling} (MCM).
These four day sprints, completed in teams of three students, emphasize pitching insights from mathematical models to business executives and policy makers.
In 2017, we developed a model of traffic in the greater Seattle area and showed that in the near future designating lanes exclusively for autonomous vehicles will reduce commuter travel delays.
We received a Finalist award, \textbf{ranking among the top 0.8\% of participating teams}.
These experiences showed me the importance of developing personal relationships and purposefully demonstrating a compelling use case for my work in the context of the target audience's perspectives and practices.%TODO

In my Sophomore year, I worked with Dr. Smith at University of Puget Sound (UPS) to \textbf{develop methods for automated isolation of mouse
ultrasonic vocalizations (USVs) from noisy recordings}.
I sought outside funding, and was awarded won NASA funding through a \textbf{competitive grant application process}.
Existing software tools identify and characterize USVs, but are often confounded by background noise.
I developed and tested filtering algorithms inspired by the Sobel Edge detection method that learn from human-annotated spectrograms distinguish between true mouse vocalization signals and background noise.
My approach achieved 75\% accuracy at 25\% recall from noisy recordings.
I presented these results at the UPS summer research symposium \cite{smith}.
I took away concrete computational research skills, including data management, version control, and visualization techniques.

I brought together my biological and computational interests together studying ant foraging behavior at the \textbf{NJIT Swarm lab}.
I was recruited by advisors Dr. Garnier and Dr. Graham through a REU coordinated by the Mathematical Biosciences Institute.
On flat terrain, the shortest-distance foraging path, the most energy-efficient path, and the quickest path are all identical.
However, on uneven terrains an obstacle may make the most direct path take longer than a trip that circumvents it.
In the absence of an absolute ``best'' path, the question of how ants make trade-offs is of great interest.
Thus, I extended computational models of ant foraging to consider uneven terrains.
My differential-equations based model predicts that severe inclines cause ants to favor a more direct, less variable foraging path.
My work culminated in presentations at the \textbf{2017 Joint Mathematics Meetings} \cite{jmm}.
During my time at the Swarm lab, I found the autonomy entrusted to me and the opportunity to answer unsolved biological questions that inform technologies like swarm robotics empowering and rewarding.
After this REU, I saw graduate school as the best opportunity to continue engaging in such self-directed, important work.

For my \textbf{senior thesis project} at UPS, I worked with Dr. Chambers to synthesize a conceptual framework to understand evolvability, the potential for adaptive change to occur in an organism's descendants.
Evolvability has been connected to a large array of causal factors.
I organized this constellation of causal factors into three larger classes and analyzed their broad relationships \cite{thesis}.
For my subsequent \textbf{capstone project}, I used a gene-regulatory network model to empirically investigate the how environmental influence on the phenotype and relates to evolvability.
I found that populations subjected to stochastic perturbations of the development process evolved a higher incidence of silent mutation and that populations evolved to respond to environmental signals by activating alternate developmental pathways were more sensitive to mutation.
For my thesis and capstone work, I received the \textbf{MacArthur Award for an Outstanding Thesis Presentation} and the \textbf{Goman Outstanding Math/CS Senior Award}.
I presented my capstone and thesis work at the 2016 NW Honors Symposium in Seattle, two campus seminars, and the 2017 BEACON Congress \cite{beacon}.
This fall, I have published my thesis work as an illustrated blog series aimed at both the general public and other scientists.
I found uniting experiments and theory in my thesis and capstone projects extremely rewarding;
this work inspired me to continue conducting research that tightly couples these domains.


\textbf{Impact}
In sixth grade, when I started to develop an interest in biology, I also began playing the oboe.
Like science, band culture is built on mentoring relationships, cooperative competition, and friendly warmth between peers.
These values are perpetuated among scientists and musicians alike by tradition and necessity.
How else can we develop extensive technical skills, push ourselves to do our best work, and accomplish goals beyond the capabilities of any single individual?
I love the willingness to invest, high expectations, and mutual respect scientists and musicians alike hold for their kind.
It was these pervasive values that inspired me to turn my interest in science into a career.
Beyond my research interests, I cherish my membership in the scientific community in and of itself.
To me, being a scientist is fundamentally entwined with furthering these values.

As an undergraduate, I built relationships with high school and college students as a subject tutor, an academic consultant, an AP tutor, a classroom assistant, a musical coach, a Q\&A panelist, and an after-school club leader.
For me, these relationships are part of what it meant to be a student.
Today, I see them among the most important accomplishments of my undergraduate career.

In particular, I remember being caught off guard by a student I was talking to about triangles at Oakland, an alternative high school.
Out of the blue, she asked, ``What's so great about college?''
In that moment, I almost launched into a biographical account of why \textit{I} liked it.
Instead, I asked, ``What's your favorite thing to talk about?''
She took a moment, then replied, ''Books, I guess.''
She elaborated on her interest for stories and characters, listing a few of her favorite pieces of young adult fiction, and bragged about reaching the cap for books that could be checked out of the library at once.
``What's so cool,'' I told her, ``is that you can take classes in whatever you want.
You can take classes about stories with other people who love them too.''
We went back and forth about \textit{really} being able to study what you want a few times before she seemed somewhat satisfied.
Before returning to triangles, I suggested that she would really get along with my humanities friends studying literature,
``That's really what they talk about.
All the time.
You'd fit in.''
She helped me realize that sharing a personal connection, a feeling, or enthusiasm is often more important than dispensing advice.
I began to recognize similar dynamics with the STEM students I tutored.
At the end of an some of my appointments, our conversation would shift to upper division coursework  and concepts that lay years ahead for them.
I took these conversations as opportunities to establish us as members of the same intellectual community instead of opportunities to dispense graduation requirements and course descriptions.
Although these were small gestures, I feel proud to have to put out the metaphorical welcome mat.

I am committed to continuing my work fostering community among scientists and performing STEM outreach.
I currently volunteer as a teacher's assistant with special education and general education classrooms in the East Lansing school district.
As I progress in my graduate studies, I look forward to mentoring undergraduate researchers and new graduate students.
In the future, I see leading a research group as an opportunity to establish the type of community I have enjoy belonging to.
I strive to follow in the footsteps of my role models and reach out to bring new members into the scientific community.
I am concerned about barriers to inclusivity in the scientific community.
My experience as a LGBT person has deepened my perspective on these barriers.
I have come to realize that identity is not baggage, to be checked at the door but must be recognized as an inherent part of belonging to the scientific community.
Although the barriers I have encountered pale in comparison to those faced by others, I hope to leverage these molehills to reach out over mountains.


As scientists, the intellectual impact of our work can be made tangible through technology.
Although my interest lies in fundamental research, by probing the algorithmic principles of biology I hope to contribute to the development of more capable and versatile artificial intelligence systems.
Beneath my own curiosity and ambition, I feel a moral imperative to participate in this work.
When reflecting on the potential human impact of advances in computing, I am reminded of my father, who volunteers with the Dial-A-Bus program in Benton county.
He serves individuals who would be otherwise unable to get around town, curtailing their ability to fully participate in the community.
I hope that, through applications that counteract disability and free us from dangerous or simply menial tasks, stronger artificial intelligence will enable people to more fully exercise their human capabilities and, thereby, lay a fuller claim to their humanity.
I see glimmers of progress towards this goal in the work I've done with the Fowler lab, the Swarm Lab, and, now, the Ofria lab.

To apply my research to real-world artificial intelligence, I will collaborate closely with industry.
I participated in such work during my my time with the Finn small fruits breeding laboratory at the USDA.
Pushing our berry cultivars out into the fresh fruit and processed food markets requires extensive and deliberate legwork.
We hosted regular symposiums around brief and actionable presentations on best practices and made a point to always wrap up with taste-tests berry sauce with ice cream.
I learned how to develop personal relationships, how to concretely explain --- and demonstrate --- a compelling use case for my work in the context of \textit{their practices}.
The scale our agricultural partners can achieve made our work at the USDA even more rewarding;
we get to see our cultivars fill vast fields that sweep over the horizon and, ultimately, in muffins, ice cream, farmers' markets, and roadside stands.
From my vantage, I see a wave of bio-inspired computing techniques sweeping through industry.
I am ready to work with industrial partners to add my research to that wave.


Ultimately, I see continuing in academic research to lead a research group as the best route to build the type of community I have enjoyed belonging to and accomplish my outreach, technological, and scientific goals.
GRFP support will help me not only conduct cutting edge research at the intersection of biology and computer science, but also to lead in the conversation about the intrinsic value of outreach, education, mentorship, and community in science.


\end{document}


%Ant colonies regulate their foraging behavior through a collective decision making process; when ants forage, no individual ant operates with complete information about the terrain they are exploring.
% This contrasts with traditional human approaches decision-making, which typically centralize information, process it, then redistribute instructions.
% Consider, for example, traffic-aware navigation tools such as Waze or Google Maps; the distribution of traffic distribution across a geographic region is collected from users, centrally processed, and then routing instructions are redistributed to individual users.
% In contrast, collective intelligence on the level of the ant colony emerges from parallel execution of a simple set of individual pheromone deposit and response behaviors; among other feats, foraging ants will tend to to choose the shortest path between nest and food and to selectively exploit the richest of an array of food sources.
