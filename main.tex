\documentclass[12pt]{book}
\usepackage[utf8]{inputenc}
\usepackage[margin=1in]{geometry}
\usepackage{amsmath}
\usepackage{graphicx}
\usepackage{setspace}
\usepackage{biblatex}
\usepackage{mathptmx}
\addbibresource{Mendeley.bib}

\title{NSF GRFP Personal Statement}
\author{Matthew Moreno}

\begin{document}
Please outline your educational and professional development plans and career goals. How do you envision graduate school preparing you for a career that allows you to contribute to expanding scientific understanding as well as broadly benefit society? Page limit - 3 pages
Describe your personal, educational and/or professional experiences that motivate your decision to pursue advanced study in science, technology, engineering or mathematics (STEM). Include specific examples of any research and/or professional activities in which you have participated. Present a concise description of the activities, highlight the results and discuss how these activities have prepared you to seek a graduate degree. Specify your role in the activity including the extent to which you worked independently and/or as part of a team. Describe the contributions of your activity to advancing knowledge in STEM fields as well as the potential for broader societal impacts (See Solicitation, Section VI, for more information about Broader Impacts).
NSF Fellows are expected to become globally engaged knowledge experts and leaders who can contribute significantly to research, education, and innovations in science and engineering. The purpose of this statement is to demonstrate your potential to satisfy this requirement. Your ideas and examples do not have to be confined necessarily to the discipline that you have chosen to pursue.
If you have completed more than 12 months of graduate or post-baccalaureate study or a professional degree and an interruption of at least two consecutive years (fourth option under Completed Study in the NSF GRFP Program Information section), please address the reasons for the interruption in graduate study here. Please refer back to that section for details.
Important questions to ask yourself before writing the statement:
\begin{enumerate}
	\item Why are you fascinated by your research area?
	\item What examples of leadership skills and unique characteristics do you bring to your chosen field?
	\item What personal and individual strengths do you have that make you a qualified applicant?
	\item How will receiving the fellowship contribute to your career goals?
	\item What are all of your applicable experiences?
	\item For each experience, what were the key questions, methodology, findings, and conclusions?
	\item Did you work in a team and/or independently?
	\item How did you assist in the analysis of results?
	\item How did your activities address the Intellectual Merit and Broader Impacts criteria?
\end{enumerate}
%\newpage

\maketitle

% \section{Self-Reflection}

Reflecting on the scientific work I have done, as well as my current interests, I see myself as straddling two divides. First, I am interested in investigating biological concepts in a computational context. A succinct self-description might be “a biologist in computer scientist’s clothing.”  Although the type of methodology I employ and, ultimately, the academic umbrella I fall under is computational, my heart lies with the architecture of biological systems. Beyond employing computational tools to directly study extant organisms or ecosystems, I am interested in understanding the fundamental design principles of biology by reverse-engineering these systems, employing those biological principles in practice. On a personal level, in addition to finding biological systems fascinating in their own right, I love the surprising ways solely material mechanisms can account for the experience of existence, a concept in which I have found an immense source of beauty and wonder. These traits account for the common thread that weaves through my scientific work and interests: fascination with emergent phenomena, systems in which properties arise which are not explicitly encoded its components, and a propensity for inquiry through building models.

My interest in reverse-engineering biological systems ties into the second disciplinary boundary I straddle: that between a scientist and an engineer. Perhaps this provides some explanation for my interest in biological systems: I see them as a window into a completely foreign philosophy of design based emergence, where properties of the system are not explicitly written into predictable internal system dynamics but instead emerge from unpredictable interactions between system components. Although I am ultimately interested in understanding questions of how? and why?, I am attracted to inquiry that builds a system to isolate the phenomenon of interest and observe it in practice. I believe that my nonconformism, the divides I straddle, set me apart from many other scientists and will continue to distinguish my work.

% \section{Research History}
%My research history begins at the Fowler Laboratory at Oregon State University, a molecular biology group studying \textit{Arabidopsis thaliana}. I spent two summers with working with my mentor Dr. Rex Cole, a research associate, searching for New Enhancers of Root Dwarfism (NERDs).
A moment of exhilarating self-awareness, I felt myself on the outset of my journey in science while examining the translucent roots of a few dozen delicate seedlings on parafilm-sealed petri dishes to look for characteristicly stumpy mutants; I relished the purpose, discipline, and inquiry of my work. Prior to my arrival at the Fowler laboratory, mutations had been chemically induced in hundreds of lines of \textit{Arabidopsis} with the \textit{sec 8-6} allele, which mildly interferes with the functionality of the exocyst protein complex. Following a five-generation set of crosses, populations derived from a self-cross of heterozygotes for \textit{sec 8-6} and mutations to putative exocyst-interacting genes were prepared for screening. Over the summer, I searched through these populations for the 13:3:1 phenotypic ratio telltale of synergistic interaction between \textit{sec 8-6} and a mutation on an independently-assorting allele. That summer, my mentor Dr. Rex Cole and I flagged two populations of putative New Enhancers of Root Dwarfism (NERDs) for further analysis. We hoped that identification of novel exocyst interactors, or confirmation of known exocyst interactors will expand and solidify understanding of biochemical processes in plants; although NERDs are not immediately applicable to technological or commercial ends, as fundamental research it is key in paving the way for other more applied advances. Since my work at the Fowler lab, I have continued to work in botany and horticulture, working with the Deluc laboratory to study the molecular biology of vineyard grape development and the USDA Finn group to develop berry cultivars for the fresh fruit and processed food markets (a position I hold to this day). 


%Most recently, I studied the foraging dynamics of ants with the Swarm Lab at the New Jersey Institute of Technology. I worked with advisors Jason Graham and Simon Garnier through a REU program organized through the Mathematical Biosciences Institute. 


Fast forward to last summer, returning to my open-office desk after a preliminary meeting with advisors Jason Graham and Simon Garnier in a swirl with surprise, adrenaline, and also a little panic. Although I was surprised and intimidated by the extent of the autonomy and discretion they entrusted to me, in retrospect I wouldn't have had it any other way; owning work was personally rewarding and, I believe, ultimately contributed to its quality. I set out to extend mathematical models of ant foraging, which are well-developed flat terrain arenas, to uneven terrains. On flat terrain, where the shortest-distance foraging path, the most energy-efficient path, and the quickest-trip path are all identical, foraging ants are known to optimize their foraging path between nest and food.
%Ant colonies regulate their foraging behavior through a collective decision making process; when ants forage, no individual ant operates with complete information about the terrain they are exploring. This contrasts with traditional human approaches decision-making, which typically centralize information, process it, then redistribute instructions. Consider, for example, traffic-aware navigation tools such as Waze or Google Maps; the distribution of traffic distribution across a geographic region is collected from users, centrally processed, and then routing instructions are redistributed to individual users. In contrast, collective intelligence on the level of the ant colony emerges from parallel execution of a simple set of individual pheromone deposit and response behaviors; among other feats, foraging ants will tend to to choose the shortest path between nest and food and to selectively exploit the richest of an array of food sources.
However, this is no longer necessarily the case on uneven terrains. Therefore, the questions of which, if any, of these paths the ants will choose to follow and, if they reliably select for a particular characteristic in their nest-to-food path, by what mechanisms this collective decision emerges from the behaviors of individual ants, are of great interest. After surveying existing models of ant foraging behavior on flat terrain and individual ant behavior on inclined surfaces, I designed and numerically evaluated an individual-based set of differential equations to model the foraging behavior of ants over uneven terrain. Analysis of time series of ant positions and orientations generated via simulation revealed that as the severity of incline the ants traverse between their nest and food increases, the model predicts ants will tend to favor a straighter and less variable foraging path. Research into the collective intelligence of insects has been directly leveraged by engineers working on swarm robotics projects, such as NASA's ant-inspired ``Swarmies'' that may one day harvest resources for Martian colonies, and distributed traffic management schemes (especially in relation to self-driving cars).

% \section{Research Interests}

Given my idiosyncratic interests, perhaps it is not surprising that I have landed in a rather niche field: evolutionary computing, generating well-adapted solutions to problems through iterative recombination and mutation of candidate solutions. I began exploring the field in a self-guided context, first stumbling into the field by dabbling in linear genetic programming. My interest in genetic programming began completely on a whim, imagining how a “truly optimal” traffic control system might be designed while waiting on an interminable red light next to my bike. The dynamics of irregular and unpredictable traffic flow between a network of lights are so complicated and nuanced that, in my estimation, a “truly” optimal (or, at least, near-optimal) system probably lies beyond the reach of traditional heuristic design approaches. A few years earlier, I had read an article on a British scientist who successfully evolved field-programmable gate array configurations to recognize a particular tone. I toyed with the idea for a few days --- developing an approach that, in retrospect, falls under the purview of linear genetic programming --- and then, after working up the courage, booted up my text editor and jumped headfirst into implementing it. 
    
Although my interest in evolutionary computation began as a hobby, I quickly fell down the rabbit hole. I had begun my work under the impression that evolutionary computation was a poorly-explored field. As I proceeded, I gradually came to realize (to my delight) that this was not the case. Seeing that the ideas I was exploring were of interest to others and encountering creative approaches to the theory and practice of evolutionary computation I had not imagined was exhilarating. Inspired by scholarly literature on genetic programming, I switched my setup to tournament-based selection (in which small subsets of individuals from a larger population are randomly chosen to compete in tournaments for survival and reproduction) and incorporated page-based recombination (employing chromosome-like structures to increase the success rate of recombination). After receiving feedback from my advisor on a written-up mock project proposal, I designed and implemented a system of concurrent “readers” to parse a linear instruction set, hoping to secure a genotype-phenotype encoding more robust to mutation and recombination. The project was an invaluable opportunity to learn and explore by playing in the sandbox, and I am so glad I seized upon this serendipitous introduction to an exciting field that, otherwise, I likely would not have discovered.

My next foray into the field of evolutionary computing was much more systematic. Standing at a whiteboard across the table from my newfound advisor, Professor Chambers, I relished her enthusiasm for exploring unfamiliar an subfield, her candid, probing approach to discussion. I distinctly felt that the stars had aligned in my favor. With her guidance, I designed a thesis project focusing on Evolving Artificial Neural Networks (EANN), a widely-explored alternative to the backpropagation method of network training. Although --- unlike backpropagation --- not requiring supervised learning during the training process and more friendly to training with recurrent structures, this approach can be stymied by its own set of challenges. In particular, designing EANN to be highly evolvable  ---  making them compatible with the evolutionary process by avoiding excessive fatal mutations and premature dead-end local maxima in the search space, for instance  ---  is a non-trivial task. Although promising research has been conducted into genotypic encodings for EANN, moving beyond direct encodings (genotypic encodings where each topological connection and weight is explicitly specified) to adaptive, implicit, and generative encodings, it is considered an open problem that has the potential to increase EANN effectiveness. My thesis project focuses on sythesizing a conceptual framework for the evolvability of EANN and empirically evaluating the effectiveness of various encoding schemes in different contexts. Although my interest lies in fundamental research, I hope to ultimately contribute to the development of more capable and versatile artificial intelligence systems.

%\section{Career Aspirations}

I am excited to continue exploring the field of evolutionary computation, in particular, meeting and collaborating with scholars already working in the field, through graduate education. Receiving the GRFP fellowship would be a major boon to my career, both in graduate school and beyond. Freeing up my time and energy from hunting down funding to support myself and from labor-intensive commitments such as teaching would directly benefit my research. More fundamentally, this funding would give me the confidence to study at institutions with smaller Computer Science departments such the University of Wyoming and the University of Central Florida that are the loci of activity in evolutionary computing. Not only would these programs align best with my interests, but they would provide the type of close-knit and self-directed community that I thrive in. I hope to use NSF GRFP funding to leverage my nontraditional background and unique perspective as a student of a liberal arts institution by lending me chachet that outstrips my non-traditional background. Finally, on a more personal level, the program's stipend funding would be a godsend as my partner and I support ourselves while he completes pharmacy school.

Beyond graduate school, my career aspiration and research philosophy are informed by the concept of novelty search. In a few words, proponents of the novelty search paradigm advocate a reduced role for an objective fitness function in evolutionary search, instead suggesting that individuals should be selected based on their novelty. Novelty search suggests that --- instead of attempting a direct (and likely deceptive) path by proceeding at each step in the direction most similar to the objective --- the best way to discover interesting elements in a search space is through a series of smaller searches in which ``stepping stones'' are collected based on their novelty value. Empirical evidence supports the validity of this approach in a technical context, but proponents of novelty search actively invite a broader interpretation of their work. The field of artificial intelligence notoriously preys upon the naivety of young students who attempt to beat a direct path towards a comprehensive solution to AI. In contrast, I am excited that I don't know the answers or even the next steps that the field will take five years down the road. I believe that only as the accumulation of modest and curiosity-driven inquiry and the collaborative exchange of stepping stones that we can arrive at an outcome of true scientific interest and social impact. My aspiration is to do exactly that.

The importance and utility of developing increasingly competent artificial intelligence by porting the design principles —-- and power -—- of biology over to computing is difficult to overstate. On a lower level than my own curiosity and ambition, on a subliminal level, I feel a moral imperative to work in this field. In my experience, the motivations of scientists --- especially those working in technology-driven fields --- can often come off as cerebral, cold, and uncompassionate. (I recall, in particular, attending a talk that closed with back-of-a-napkin calculation of the limit imposed by relativistic constraints on the maximum number of human lives we could create over the lifespan of the universe). I strive to check this tendency in myself by exercising self-awareness and striving to engage directly with causes I believe in. Nevertheless, I aspire to contribute to lofty, perhaps abstract, goals as well. While material standards of living and metrics measuring physical health are important, they are just part of the puzzle; I hope stronger artificial intelligence will free people from dangerous, as well as simply menial, tasks such as commuting via automobile or household management duties to more fully exercise their human capabilities and, thereby, lay a fuller claim to their humanity. In pursuit of social, personal, and scientific impact, I am excited to jump into bio-inspired approaches to artificial intelligence; the NSF GRFP would allow me to hit the ground running.

\end{document}