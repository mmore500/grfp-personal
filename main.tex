\documentclass[12pt]{book}
\usepackage[utf8]{inputenc}
\usepackage[margin=1in]{geometry}
\usepackage{amsmath}
\usepackage{graphicx}
\usepackage{setspace}
\usepackage{biblatex}
\usepackage{mathptmx}
\addbibresource{Mendeley.bib}

\renewcommand{\rmdefault}{ptm}

\title{NSF GRFP Personal Statement}
\author{Matthew Moreno}
\pagenumbering{gobble}

\begin{document}
Please outline your educational and professional development plans and career goals. How do you envision graduate school preparing you for a career that allows you to contribute to expanding scientific understanding as well as broadly benefit society? Page limit - 3 pages
Describe your personal, educational and/or professional experiences that motivate your decision to pursue advanced study in science, technology, engineering or mathematics (STEM). Include specific examples of any research and/or professional activities in which you have participated. Present a concise description of the activities, highlight the results and discuss how these activities have prepared you to seek a graduate degree. Specify your role in the activity including the extent to which you worked independently and/or as part of a team. Describe the contributions of your activity to advancing knowledge in STEM fields as well as the potential for broader societal impacts (See Solicitation, Section VI, for more information about Broader Impacts).
NSF Fellows are expected to become globally engaged knowledge experts and leaders who can contribute significantly to research, education, and innovations in science and engineering. The purpose of this statement is to demonstrate your potential to satisfy this requirement. Your ideas and examples do not have to be confined necessarily to the discipline that you have chosen to pursue.
If you have completed more than 12 months of graduate or post-baccalaureate study or a professional degree and an interruption of at least two consecutive years (fourth option under Completed Study in the NSF GRFP Program Information section), please address the reasons for the interruption in graduate study here. Please refer back to that section for details.
Important questions to ask yourself before writing the statement:
\begin{enumerate}
	\item Why are you fascinated by your research area?
	\item What examples of leadership skills and unique characteristics do you bring to your chosen field?
	\item What personal and individual strengths do you have that make you a qualified applicant?
	\item How will receiving the fellowship contribute to your career goals?
	\item What are all of your applicable experiences?
	\item For each experience, what were the key questions, methodology, findings, and conclusions?
	\item Did you work in a team and/or independently?
	\item How did you assist in the analysis of results?
	\item How did your activities address the Intellectual Merit and Broader Impacts criteria?
\end{enumerate}
%\newpage

\maketitle

%\section{Introduction/Self-Reflection}

My scientific interests and work straddles disciplinary boundaries between biology and computing and, more broadly, between engineering and science. I love both to explore and to create. From many years of playing the oboe, I would describe science and engineering like studying the theory and history of a piece and performing it; exploring and creating inform each other and, together, are electric. In sixth grade, right while I was starting off on the oboe, I also got my start in biology. I remember hot gluing a collection of tiny beads to my cardboard cell model, struggling to deduce how the rather simplistic descriptions of organelles we had learned could account for the entirety of a cohesive system. My first reaction was disbelief that a system governed by random interactions between innumerable independent components could function. In the same breath, though, I knew it must be possible; evidence --- and, ultimately, the answers to the question how? --- permeates our living world. From that point onward, the surprises embedded at every level of biological organization have continuously delighted me; studying and researching biology is exploration in the truest, most adventurous, sense of the word. The architecture of biological systems is a window into the completely foreign paradigm of emergence where, instead of a clockwork of predictable internal system dynamics, properties of a system instead emerge from interactions between independent components. My career aspiration is in the field bio-AI, studying and harnessing biologically-inspired approaches for the design of intelligent systems. I believe that the divides I straddle, between biology and computing and --- ultimately --- between engineering and science, set me apart and will continue to distinguish my work.

%\section{Research History}
My first foray into research was with the Fowler Laboratory at Oregon State University. I recall, in particular, from my summer there a moment --- examining the translucent roots of a few dozen delicate seedlings on parafilm-sealed petri dishes to look for characteristically stumpy mutants --- of exhilarating self-awareness: knowing that I was at the outset of my journey in science; I relished the purpose, discipline, and inquiry of my work. I set out to identify populations of \textit{Arabidopsis thaliana} with mutations to exocyst-interacting alleles by screening for populations with a 13:3:1 phenotypic ratio telltale of synergistic interaction between \textit{sec 8-6} and a mildly deleterious mutation at an independently-assorting locus. After performing a statistical analysis of my observations, I flagged two populations of putative New Enhancers of Root Dwarfism (NERDs) for further analysis. My mentor, Dr. Cole, and I hoped that identification of novel exocyst interactors, or confirmation of known exocyst interactors, would expand and solidify understanding of role of the exocyst in plants; although NERD identification did not immediately translate to technological or commercial ends, as fundamental research it is key to opening further avenues of inquiry for future study and, ultimately, in paving the way for other more applied advances. Since my time at the Fowler lab, I have continued to work in botany and horticulture, working with the Deluc laboratory to study the molecular biology of vineyard grape development and the USDA Finn group to develop berry cultivars for the fresh fruit and processed food markets. From my work in laboratory biology, I became versed in experimentally-driven inquiry --- as well as the carefully-considered design and methodological meticulousness it requires --- which, I believe, lends me unique perspective in the domains of computer science and mathematics and which I look forward to continuing to develop in graduate school.


Fast forward to my work with the Swarm Lab at NJIT last summer. I sat at my new desk in the open-plan office after a preliminary meeting with advisors Jason Graham and Simon Garnier, consumed by a swirl of surprise, adrenaline, and, also, a little panic. The meeting boiled down to this: ``Here's the problem. Here are some resources. Take a crack at it.'' Although I was initially surprised and intimidated by the extent of the autonomy and discretion they entrusted to me, in retrospect I wouldn't have had it any other way; owning my work was personally rewarding and, I believe, ultimately contributed to its quality. My task to extend mathematical models of ant foraging, which are well-developed on flat surfaces, to uneven terrains. It is well known that, as a collective, ants can optimize the foraging path they travel between nest and food. On flat terrain, a clear ``best'' path exists: the shortest-distance foraging path, the most energy-efficient path, and the quickest-trip path are all identical. On uneven terrains, however, this is no longer necessarily the case and the question of which trade-offs ants make --- and how they make them --- is of great interest. After surveying existing models of ant foraging behavior on flat terrain and individual ant behavior on inclined surfaces, I designed and numerically evaluated an differential equations-based model of the foraging behavior of ants over uneven terrain. Analysis of time series of ant positions and orientations generated via simulation revealed that as the severity of the incline the ants traverse increases, the model predicts ants will tend to favor more direct and less variable foraging path. My model and results, in tandem with upcoming in vivo ant foraging experiments at the Swarm Lab, will help to address broader questions of which paths the ants choose through uneven terrain and by what mechanisms this collective decision emerges from the behaviors of individual ants. Ultimately, research into the collective intelligence of insects translates directly to applications in technology; for example, such research has been leveraged in swarm robotics projects, such as NASA's ant-inspired ``Swarmies'' that may one day harvest resources for Martian colonies, and distributed traffic management schemes (especially in relation to self-driving cars). On a personal level, my research experience has translated into a penchant for autonomously chewing on difficult problems, which I hope will serve me well throughout graduate school.

%\section{Research Interests}

Given my idiosyncratic interests, perhaps it is not surprising that I have landed in a rather niche field: evolutionary computing, which aims to generate well-adapted solutions to problems through iterative recombination and mutation of candidate solutions. I stumbled into the field while waiting on an interminable red light next to my bike. I reasoned that the dynamics of irregular and unpredictable traffic flow between a network of lights were so complicated and nuanced that an optimal (or, at least, near-optimal) system lies beyond the reach of traditional heuristic design approaches. A memory surfaced of an article about the evolution of field-programmable gate array configurations to recognize a particular tone. I toyed with the idea for a few days --- developing an approach that, in retrospect, falls under the purview of linear genetic programming --- and then, after working up the courage, booted up my text editor and jumped headfirst into implementing it. 

Although my interest in evolutionary computing began as a hobby, in short order I fell down the rabbit hole. Seeing that the ideas I was exploring were of interest to others and encountering creative approaches to the theory and practice of evolutionary computation I had not imagined was exhilarating. Inspired by scholarly literature on genetic programming, I switched my setup to tournament-based selection and incorporated page-based recombination. After receiving feedback from my advisor on a mock project proposal, I designed and implemented a system of concurrent ``readers'' to parse a linear instruction set, hoping to secure a genotype-phenotype encoding more robust to mutation and recombination. The project was an invaluable opportunity to learn and explore by playing in the sandbox, and I am so glad I seized upon this serendipitous introduction to an exciting field that, otherwise, I likely would not have discovered.

My next foray into the field of evolutionary computing was much more systematic. Standing at a whiteboard across the table from my newfound advisor, Professor Chambers, her keenness to explore unfamiliar theory and methods in evolutionary computing and her candid, probing approach to discussion buoyed my own enthusiasm; I distinctly felt that the stars had aligned in my favor. With her guidance, I designed a thesis project focusing on Evolving Artificial Neural Networks (EANN), a widely-explored alternative to the backpropagation method of network training. Designing EANN to be highly evolvable  ---  promoting the generation of useful variation during the evolutionary process  ---  is a non-trivial task. Promising research has investigated the roles of genetic encodings for EANN and selective pressure in promoting evolvability. My thesis project focuses on sythesizing a conceptual framework for the evolvability of EANN and empirically investigating the role of developmental plasticity and environmental perturbation in promoting evolvability. Although my interest lies in fundamental research, I hope to ultimately contribute to the development of more capable and versatile artificial intelligence systems.

%\section{Career Aspirations}
The importance and utility of developing increasingly competent artificial intelligence by porting the design principles —-- and power -—- of biology over to computing is difficult to overstate. Below my own curiosity and ambition, on a subliminal level, I feel a moral imperative to work in the field of bio-AI. In my experience, the motivations of scientists --- especially those working in technology-driven fields --- can often come off as cerebral and abstract. (I recall, in particular, attending a talk that closed with back-of-a-napkin calculation of the limit imposed by relativistic constraints on the maximum number of human lives we could create over the lifespan of the universe). I strive to check this tendency in myself by exercising self-awareness and engaging directly with causes I believe in, such as youth outreach. I am reminded of my father, who volunteers with the Dial-A-Bus program in Benton county to serve individuals who would be otherwise unable to get around town, severely curtailing their ability to participate fully in the community. I hope that, through applications such as autonomous vehicles that counteract disability and free us from dangerous or simply menial tasks, stronger artificial intelligence will enable people to more fully exercise their human capabilities and, thereby, lay a fuller claim to their humanity. The potential that bio-AI offers on this front, in my opinion, is unparalleled. On a more existential level, I wish to give back to the scientific account of the how material mechanisms underpin the experience of existence, in which I have found an immense, austere beauty; bio-AI provides an unparalleled path to this end, as well.

%\subsection{How NSF GRFP funding would help}

I am excited to continue exploring the field of evolutionary computation through graduate education, an experience which GRFP funding would turbocharge. Beyond freeing up my time to focus on research, it would allow me to study at institutions with smaller Computer Science departments where funding might be less certain such the University of Wyoming and the University of Central Florida that are forerunners in the field of evolutionary computing. Not only would these programs align best with my interests, but they would provide the type of close-knit and self-directed community that I thrive in. 

My ultimate career aspiration is to channel my graduate education into a career in industry. In just the last few years there has been a ground surge of bio-AI activity in the technology sector; large technology companies such as Google, Facebook, and IBM have jumped into the field and new players that specialize in bio-AI, such as evolutionary computing-focused Sentient Technologies, have come to the table. The results that this investment has already yielded, such as AlphaGo's besting human play at the notoriously difficult game of Go or human-level image recognition, are already very concrete, numerous, and --- moreover --- exciting. The electrifying potential in industry is due, in part, to the unparalleled resources it can muster --- in terms of computational power, data, but also in terms of people; industry hosts a thriving and collaborative community of thought-leaders in AI, a group which I plan to join. Further, industry provides a unique outlet to connect theory and application; I believe that bio-AI advances will accumulate through mutual catalysis between incremental advances in theory and applied work. I am excited to jump into bio-inspired approaches to artificial intelligence; the NSF GRFP would allow me to hit the ground running.

\end{document}

%Ant colonies regulate their foraging behavior through a collective decision making process; when ants forage, no individual ant operates with complete information about the terrain they are exploring. This contrasts with traditional human approaches decision-making, which typically centralize information, process it, then redistribute instructions. Consider, for example, traffic-aware navigation tools such as Waze or Google Maps; the distribution of traffic distribution across a geographic region is collected from users, centrally processed, and then routing instructions are redistributed to individual users. In contrast, collective intelligence on the level of the ant colony emerges from parallel execution of a simple set of individual pheromone deposit and response behaviors; among other feats, foraging ants will tend to to choose the shortest path between nest and food and to selectively exploit the richest of an array of food sources.