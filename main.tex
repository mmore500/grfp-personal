\documentclass[12pt]{book}
\usepackage[utf8]{inputenc}
\usepackage[margin=1in]{geometry}
\usepackage{amsmath}
\usepackage{graphicx}
\usepackage{setspace}
\usepackage{biblatex}
\usepackage{mathptmx}
\addbibresource{Mendeley.bib}

\renewcommand{\rmdefault}{ptm}

\title{NSF GRFP Personal Statement}
\author{Matthew Moreno}
\pagenumbering{gobble}

\begin{document}

In sixth grade, I was up late to glue tiny beads from my family's junk drawer onto the cardboard cell model that I had traced out.
While I struggled to place the small plastic pieces onto my model, I also struggled to link the simplistic descriptions of organelles we had learned into an understanding of the cellular whole.
Because every piece of the cell depends on others, I felt I could not truly understand any piece in isolation.
That experience stoked a glimmer of curiosity that grew into a love of thinking about and experimenting with abstractions of the \textbf{astonishing complex structures and mechanisms embedded at every level of biological organization}.
Today, \textbf{computational experiments} are my vehicle for this endeavor.

When I started to develop an interest in biology, I also began playing the oboe.
Like science, band culture is built on \textbf{mentoring relationships, cooperative competition, and friendly warmth between peers}.
Without this community, we could not develop extensive technical skills, push ourselves to do our best work, and accomplish goals far beyond the capabilities of any single individual.
The willingness to invest time and energy, high expectations, and mutual respect scientists hold for their kind inspired me to pursue a career in science.

As a scientist, I aspire
(1) to \textbf{develop fundamental theory and applications for bio-inspired approaches to AI} and
(2) to \textbf{promote an inclusive and intellectually vibrant science community through mentoring relationships and STEM education}.


\textbf{Research Interests}
I am interested in understanding what critical nuances are missing from simple algorithmic implementations of evolution built on bare bones selection, variation, and inheritance.
As an undergraduate advised by Dr. America Chambers, I synthesized a conceptual framework to understand evolvability, the potential for adaptive change to occur in an organism's descendants.
The origins of evolvability have been connected to a large array of causal factors.
I organized this constellation of causal factors into three larger classes and analyzed their broad relationships.
For my subsequent capstone project, I used a gene-regulatory network model to empirically investigate the relationship between environmental influence on the phenotype and evolvability.
I found that populations subjected to stochastic perturbations to the development process evolved a higher incidence of silent mutation.
I hypothesized that disruptive environmental influence on the phenotype selected against unstable cyclic regulatory interactions that otherwise would have remained invisible, thus increasing the frequency at which mutations had no phenotypic effect.
Working with the same model, I also found that organisms evolved to respond to environmental signals by activating alternate developmental pathways were more sensitive to mutation.

My research contributes to ongoing scientific conversations about evolvability.
Evolutionary biologists see evolvability as central to understanding the evolution of complex traits such as human intelligence.
Questions about the role of environmental influence on the phenotype in the evolution of novel traits, in particular, are central to debate over recent, controversial extensions to the canonical modern evolutionary synthesis.
In turn, evolutionary computing researchers seek more evolvable systems to solve practical problems.
Translating more nuanced elements of the evolutionary process like environmental influence on the phenotype will yield powerful applied evolution techniques.
I am a doctoral student at Michigan State University, working with Dr. Charles Ofria in the NSF BEACON Center for the study of evolution in action.
I am uniquely positioned to contribute to conversations about evolutionary dynamics to apply novel biological concepts to evolutionary computation.


\textit{\textbf{Research History}}
I began gaining research experience in high school.
Through a local \textbf{Apprenticeships in Science program}, I investigated the exocyst complex in plants with the Fowler lab at Oregon State University.
I used genetic assays and phenotypic measurements to screen 25 populations of \textit{Arabadopsis thaliana} for synergistic interaction between a known exocyst mutation and a mildly deleterious mutation at an independently-assorting locus.
The group went on to isolate a novel Golgi-localized protein confirmed to interact with the exocyst \cite{fowler}.
I took away a strong impression of what a friendly, supportive, and inclusive scientific community looks like.
This personal experience with STEM outreach motivates my desire to make similar opportunities available to others.

Beginning my college career, I worked two full-time summers with the \textbf{USDA small fruits breeding laboratory}.
The group, led by Dr. Chad Finn, develops berry cultivars for the fresh fruit and processed food markets.
Realizing widespread production of those cultivars requires deliberate cooperation with partner growers.
Among other efforts, we hosted regular symposiums around brief and actionable presentations on best practices and made a point to always wrap up with taste-tests of berry sauce with ice cream.
Later in my undergraduate career, I exercised my experience at the interface of science and industry through three bouts in the \textbf{Mathematical Competition in Modeling} (MCM).
These four day sprints, completed in teams of three students, emphasize pitching insights from mathematical models to business executives and policy makers.
In 2017, we developed a model of traffic in the greater Seattle area and showed that in the near future designating lanes exclusively for autonomous vehicles will reduce commuter travel delays.
We received a Finalist award, \textbf{ranking among the top 0.8\% of participating teams}.
These experiences showed me the importance of developing personal relationships and purposefully demonstrating a compelling use case for my work in the context of the target audience's perspectives and practices.%TODO

In my Sophomore year, I worked with Dr. Smith at University of Puget Sound (UPS) to \textbf{develop methods for automated isolation of mouse
ultrasonic vocalizations (USVs) from noisy recordings}.
I sought outside funding, and was awarded won NASA funding through a \textbf{competitive grant application process}.
Existing software tools identify and characterize USVs, but are often confounded by background noise.
I developed and tested filtering algorithms inspired by the Sobel Edge detection method that learn from human-annotated spectrograms distinguish between true mouse vocalization signals and background noise.
My approach achieved 75\% accuracy at 25\% recall from noisy recordings.
I presented these results at the UPS summer research symposium \cite{smith}.
I took away concrete computational research skills, including data management, version control, and visualization techniques.

I brought together my biological and computational interests together studying ant foraging behavior at the \textbf{NJIT Swarm lab}.
I was recruited by advisors Dr. Garnier and Dr. Graham through a REU coordinated by the Mathematical Biosciences Institute.
On flat terrain, the shortest-distance foraging path, the most energy-efficient path, and the quickest path are all identical.
However, on uneven terrains an obstacle may make the most direct path take longer than a trip that circumvents it.
In the absence of an absolute ``best'' path, the question of how ants make trade-offs is of great interest.
Thus, I extended computational models of ant foraging to consider uneven terrains.
My differential-equations based model predicts that severe inclines cause ants to favor a more direct, less variable foraging path.
My work culminated in presentations at the \textbf{2017 Joint Mathematics Meetings} \cite{jmm}.
During my time at the Swarm lab, I found the autonomy entrusted to me and the opportunity to answer unsolved biological questions that inform technologies like swarm robotics empowering and rewarding.
After this REU, I saw graduate school as the best opportunity to continue engaging in such self-directed, important work.

For my \textbf{senior thesis project} at UPS, I worked with Dr. Chambers to synthesize a conceptual framework to understand evolvability, the potential for adaptive change to occur in an organism's descendants.
Evolvability has been connected to a large array of causal factors.
I organized this constellation of causal factors into three larger classes and analyzed their broad relationships \cite{thesis}.
For my subsequent \textbf{capstone project}, I used a gene-regulatory network model to empirically investigate the how environmental influence on the phenotype and relates to evolvability.
I found that populations subjected to stochastic perturbations of the development process evolved a higher incidence of silent mutation and that populations evolved to respond to environmental signals by activating alternate developmental pathways were more sensitive to mutation.
For my thesis and capstone work, I received the \textbf{MacArthur Award for an Outstanding Thesis Presentation} and the \textbf{Goman Outstanding Math/CS Senior Award}.
I presented my capstone and thesis work at the 2016 NW Honors Symposium in Seattle, two campus seminars, and the 2017 BEACON Congress \cite{beacon}.
This fall, I have published my thesis work as an illustrated blog series aimed at both the general public and other scientists.
I found uniting experiments and theory in my thesis and capstone projects extremely rewarding;
this work inspired me to continue conducting research that tightly couples these domains.


\textbf{Impact}
In sixth grade, when I started to develop an interest in biology, I also began playing the oboe.
Like science, band culture is built on mentoring relationships, cooperative competition, and friendly warmth between peers.
These values are perpetuated among scientists and musicians alike by tradition and necessity.
How else can we develop extensive technical skills, push ourselves to do our best work, and accomplish goals beyond the capabilities of any single individual?
I love the willingness to invest, high expectations, and mutual respect scientists and musicians alike hold for their kind.
It was these pervasive values that inspired me to turn my interest in science into a career.
Beyond my research interests, I cherish my membership in the scientific community in and of itself.
To me, being a scientist is fundamentally entwined with furthering these values.

As an undergraduate, I built relationships with high school and college students as a subject tutor, an academic consultant, an AP tutor, a classroom assistant, a musical coach, a Q\&A panelist, and an after-school club leader.
For me, these relationships are part of what it meant to be a student.
Today, I see them among the most important accomplishments of my undergraduate career.

In particular, I remember being caught off guard by a student I was talking to about triangles at Oakland, an alternative high school.
Out of the blue, she asked, ``What's so great about college?''
In that moment, I almost launched into a biographical account of why \textit{I} liked it.
Instead, I asked, ``What's your favorite thing to talk about?''
She took a moment, then replied, ''Books, I guess.''
She elaborated on her interest for stories and characters, listing a few of her favorite pieces of young adult fiction, and bragged about reaching the cap for books that could be checked out of the library at once.
``What's so cool,'' I told her, ``is that you can take classes in whatever you want.
You can take classes about stories with other people who love them too.''
We went back and forth about \textit{really} being able to study what you want a few times before she seemed somewhat satisfied.
Before returning to triangles, I suggested that she would really get along with my humanities friends studying literature,
``That's really what they talk about.
All the time.
You'd fit in.''
She helped me realize that sharing a personal connection, a feeling, or enthusiasm is often more important than dispensing advice.
I began to recognize similar dynamics with the STEM students I tutored.
At the end of an some of my appointments, our conversation would shift to upper division coursework  and concepts that lay years ahead for them.
I took these conversations as opportunities to establish us as members of the same intellectual community instead of opportunities to dispense graduation requirements and course descriptions.
Although these were small gestures, I feel proud to have to put out the metaphorical welcome mat.

I am committed to continuing my work fostering community among scientists and performing STEM outreach.
I currently volunteer as a teacher's assistant with special education and general education classrooms in the East Lansing school district.
As I progress in my graduate studies, I look forward to mentoring undergraduate researchers and new graduate students.
In the future, I see leading a research group as an opportunity to establish the type of community I have enjoy belonging to.
I strive to follow in the footsteps of my role models and reach out to bring new members into the scientific community.
I am concerned about barriers to inclusivity in the scientific community.
My experience as a LGBT person has deepened my perspective on these barriers.
I have come to realize that identity is not baggage, to be checked at the door but must be recognized as an inherent part of belonging to the scientific community.
Although the barriers I have encountered pale in comparison to those faced by others, I hope to leverage these molehills to reach out over mountains.


As scientists, the intellectual impact of our work can be made tangible through technology.
Although my interest lies in fundamental research, by probing the algorithmic principles of biology I hope to contribute to the development of more capable and versatile artificial intelligence systems.
Beneath my own curiosity and ambition, I feel a moral imperative to participate in this work.
When reflecting on the potential human impact of advances in computing, I am reminded of my father, who volunteers with the Dial-A-Bus program in Benton county.
He serves individuals who would be otherwise unable to get around town, curtailing their ability to fully participate in the community.
I hope that, through applications that counteract disability and free us from dangerous or simply menial tasks, stronger artificial intelligence will enable people to more fully exercise their human capabilities and, thereby, lay a fuller claim to their humanity.
I see glimmers of progress towards this goal in the work I've done with the Fowler lab, the Swarm Lab, and, now, the Ofria lab.

To apply my research to real-world artificial intelligence, I will collaborate closely with industry.
I participated in such work during my my time with the Finn small fruits breeding laboratory at the USDA.
Pushing our berry cultivars out into the fresh fruit and processed food markets requires extensive and deliberate legwork.
We hosted regular symposiums around brief and actionable presentations on best practices and made a point to always wrap up with taste-tests berry sauce with ice cream.
I learned how to develop personal relationships, how to concretely explain --- and demonstrate --- a compelling use case for my work in the context of \textit{their practices}.
The scale our agricultural partners can achieve made our work at the USDA even more rewarding;
we get to see our cultivars fill vast fields that sweep over the horizon and, ultimately, in muffins, ice cream, farmers' markets, and roadside stands.
From my vantage, I see a wave of bio-inspired computing techniques sweeping through industry.
I am ready to work with industrial partners to add my research to that wave.


Ultimately, I see continuing in academic research to lead a research group as the best route to build the type of community I have enjoyed belonging to and accomplish my outreach, technological, and scientific goals.
GRFP support will help me not only conduct cutting edge research at the intersection of biology and computer science, but also to lead in the conversation about the intrinsic value of outreach, education, mentorship, and community in science.


\end{document}


%Ant colonies regulate their foraging behavior through a collective decision making process; when ants forage, no individual ant operates with complete information about the terrain they are exploring.
% This contrasts with traditional human approaches decision-making, which typically centralize information, process it, then redistribute instructions.
% Consider, for example, traffic-aware navigation tools such as Waze or Google Maps; the distribution of traffic distribution across a geographic region is collected from users, centrally processed, and then routing instructions are redistributed to individual users.
% In contrast, collective intelligence on the level of the ant colony emerges from parallel execution of a simple set of individual pheromone deposit and response behaviors; among other feats, foraging ants will tend to to choose the shortest path between nest and food and to selectively exploit the richest of an array of food sources.
