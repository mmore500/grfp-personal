\textit{\textbf{Outreach History}}
As an undergraduate, I built relationships with high school and college students as \textbf{a subject tutor, an academic consultant, an AP tutor, a classroom assistant, a musical coach, and an after-school club leader}.
In these capacities, I worked with high school students on a weekly basis over two and a half years and with other college students on a weekly basis for three years.
Today, I see my relationships with other students among the most important accomplishments of my undergraduate career.

In particular, I remember being caught off guard by a student I was talking to about triangles at Oakland, an alternative high school.
Out of the blue, she asked, ``What's so great about college?''
In that moment, I almost launched into a biographical account of why \textit{I} liked it.
Instead, I asked, ``What's your favorite thing to talk about?''
She took a moment, then replied, ''Books, I guess.''
She elaborated on her favorite for stories and characters and bragged about reaching the library checkout cap.
``What's so cool,'' I told her, ``is that you can take classes in whatever you want.
You can take classes about stories with other people who love them too.''
We went back and forth about \textit{really} being able to study what you want a few times before she seemed somewhat satisfied.
Before returning to triangles, I suggested that she would really get along with my humanities friends,
``That's really what they talk about.
All the time.
You'd fit in.''
She helped me realize that \textbf{making a personal connection and sharing my enthusiasm is more important than dispensing advice}.

I began to recognize similar dynamics with the college-age STEM students I tutored.
These conversations inspired me to \textbf{lead a computer science departmental mentoring initiative in my senior year}.
I recruited upperclassmen mentors to lead small groups of underclassmen in computer science activities, emphasizing social connections between class cohorts and important skills not explicitly taught in the classroom like professional networking and integrated development environments.
We organized three meet-ups, each attended by approximately fifteen students.
Although these were small gestures, I feel proud to have to put out the metaphorical welcome mat.
