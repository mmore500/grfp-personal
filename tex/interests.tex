%\section{Research Interests}
%
% I stumbled into my current research interests while waiting at an interminable red light on my bike four years ago.
% I reasoned that the dynamics of traffic flow through a network of irregular intersections are so complicated and nuanced that an optimal (or, at least, near-optimal) system lies beyond the reach of traditional heuristic design approaches.
% An article I had encountered years prior surfaced in my memory.
% It documented efforts to generate field-programmable gate array configurations well-adapted to certain tasks through repeated selection, recombination, and mutation of candidate configurations.
% This type of work is part of the larger field of evolutionary computation, which aims to employ algorithms inspired by evolution for problem-solving ends.
%
% I toyed with the problem for a few days --- developing an approach that, in retrospect, falls under the purview of linear genetic programming --- and then, after working up the courage, booted up my text editor and jumped headfirst into implementing it.
% Although my interest in evolutionary computing began as a hobby, in short order I fell down the rabbit hole.
% It was exhilarating to see that the ideas I was exploring were of interest to others.
% Inspired by scholarly literature on genetic programming, I switched my evolutionary algorithm to tournament-based selection and incorporated page-based recombination.
%
% Although traffic control had originally sparked my interest, I quickly switched to working in other, more standard, problem domains such as pole balancing and maze exploration.
% Receiving guidance from my undergraduate advisor Professor Adam Smith, with whom I had developed tools to extract mouse vocalizations from noisy recordings during the summer of 2015, I wrote a mock project proposal laying out the design for a system of concurrent instruction readers to parse a linear instruction set.
% The aim of this scheme was to secure a genotype-phenotype encoding more robust to mutation and recombination.
% Carrying out this project was an invaluable opportunity to learn and explore by playing in the sandbox.

My current interest is in using computational models to better understand evolution.
In this realm, understanding evolvability  ---  promoting the generation of useful variation during the evolutionary process  ---  is a difficult, but important, task.
Investigating evolvability amounts to digging into the relationship between the configuration of a system and the outcomes of mutational perturbation to that system.
My senior thesis, advised by Professor America Chambers, focused on synthesizing a conceptual framework for evolvability.
My subsequent capstone project investigated the relationship between environmental influence on the phenotype and evolvability.
In experiments performed with a gene regulatory network model, I found that populations evolved under a regime stochastic environmental perturbation of the developmental process were more resistant to mutation.
Using the same model, I also found that populations evolved under a selective pressure to respond to environmental signals by activating alternate developmental pathways were more sensitive to mutation.
I hypothesize that, in these cases, environmental influence on the phenotype led to selection for certain internal structural characteristics that modulate the phenotypic consequences of that environmental influence.
These internal structural characteristics, in turn, affect mutational outcomes.
Probing the question of evolvability has immediate implications to evolutionary computing, potentially yielding more powerful digital evolution techniques, in addition to relevance to the field of evolutionary biology, which is grappling with the relationship between environmental influence on the phenotype and evolution.
