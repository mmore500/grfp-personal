%\section{Research Interests}
%
% I stumbled into my current research interests while waiting at an interminable red light on my bike four years ago.
% I reasoned that the dynamics of traffic flow through a network of irregular intersections are so complicated and nuanced that an optimal (or, at least, near-optimal) system lies beyond the reach of traditional heuristic design approaches.
% An article I had encountered years prior surfaced in my memory.
% It documented efforts to generate field-programmable gate array configurations well-adapted to certain tasks through repeated selection, recombination, and mutation of candidate configurations.
% This type of work is part of the larger field of evolutionary computation, which aims to employ algorithms inspired by evolution for problem-solving ends.
%
% I toyed with the problem for a few days --- developing an approach that, in retrospect, falls under the purview of linear genetic programming --- and then, after working up the courage, booted up my text editor and jumped headfirst into implementing it.
% Although my interest in evolutionary computing began as a hobby, in short order I fell down the rabbit hole.
% It was exhilarating to see that the ideas I was exploring were of interest to others.
% Inspired by scholarly literature on genetic programming, I switched my evolutionary algorithm to tournament-based selection and incorporated page-based recombination.
%
% Although traffic control had originally sparked my interest, I quickly switched to working in other, more standard, problem domains such as pole balancing and maze exploration.
% Receiving guidance from my undergraduate advisor Professor Adam Smith, with whom I had developed tools to extract mouse vocalizations from noisy recordings during the summer of 2015, I wrote a mock project proposal laying out the design for a system of concurrent instruction readers to parse a linear instruction set.
% The aim of this scheme was to secure a genotype-phenotype encoding more robust to mutation and recombination.
% Carrying out this project was an invaluable opportunity to learn and explore by playing in the sandbox.

My current interest is in using computational models to better understand evolution.
Working with advisor Dr. America Chambers, I synthesized a conceptual framework to understand evolvability as my undergraduate thesis work.
Evolvability, the relationship between the configuration of a system and the phenotypic outcomes of mutational perturbation to that system, has been connected to an overwhelmingly large array of causal factors.
I organized this constellation of causal factors into three larger classes and described broader relationships between these classes.
My subsequent capstone project empirically investigated the relationship between environmental influence on the phenotype and evolvability.
In experiments performed with a gene regulatory network model, I found that populations evolved with stochastic perturbations to the development process had a higher incidence of silent mutation.
I hypothesize that modeling disruptive environmental influence on the phenotype introduced selection on internal structural characteristics --- specifically, against unstable cyclic regulatory interactions --- that otherwise would have remained invisible.
These internal structural characteristics, in turn, affect the phenotypic outcomes observed under mutation, in this case increasing the frequency at which mutations had no phenotypic effect.
Working with the same model, I also found that populations evolved under a selective pressure to respond to environmental signals by activating alternate developmental pathways were more sensitive to mutation.
My capstone and thesis work contribute to active conversations about evolvability taking place among and between evolutionary computing and evolutionary biology researchers.
Evolvability has immediate practical implications to evolutionary computing;
increasing the novelty and usefulness of mutational outcomes is key to developing more powerful digital evolution techniques.
Evolutionary biology, in turn, is grappling with a number of recent, controversial extensions to the canonical modern synthesis.
Key to evaluating these extensions are questions of the evolutionary origins of evolvability and the role of environmental influence on the phenotype in the evolution of novel traits.
