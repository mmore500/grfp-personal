%\section{Research Interests}

My research interests lie in a rather niche field: evolutionary computing, which aims to generate well-adapted solutions to problems through iterative recombination and mutation of candidate solutions.
I stumbled into the field while waiting at an interminable red light on my bike.
I reasoned that the dynamics of traffic flow through a network of irregular intersections are so complicated and nuanced that an optimal (or, at least, near-optimal) system lies beyond the reach of traditional heuristic design approaches.
An article I had encountered years prior about performing evolution on field programmable gate arrays surfaced in my memory.
I toyed with the problem for a few days --- developing an approach that, in retrospect, falls under the purview of linear genetic programming --- and then, after working up the courage, booted up my text editor and jumped headfirst into implementing it.
Although my interest in evolutionary computing began as a hobby, in short order I fell down the rabbit hole.
It was exhilarating to see that the ideas I was exploring were of interest to others.
Inspired by scholarly literature on genetic programming, I switched my evolutionary algorithm to tournament-based selection and incorporated page-based recombination.
Although traffic control had originally sparked my interest, I quickly switched to working in other, more standard, problem domains such as pole balancing and maze exploration.
Receiving guidance from my undergraduate advisor Professor Smith, with whom I had developed tools to extract mouse vocalizations from noisy recordings during the summer of 2015, I wrote a mock project proposal laying out the design for a system of concurrent instruction readers to parse a linear instruction set.
The aim of this scheme was to secure a genotype-phenotype encoding more robust to mutation and recombination.
Carrying out this project was an invaluable opportunity to learn and explore by playing in the sandbox.

My next foray into the field of evolutionary computing has been much more systematic.
This year, I have been working with Professor Chambers on my senior thesis.
With her guidance, I designed a thesis project exploring Evolving Artificial Neural Networks (EANNs), a widely-explored alternative to the backpropagation method of network training.
Designing EANNs to be highly evolvable  ---  promoting the generation of useful variation during the evolutionary process  ---  is a difficult, but important, task.
My thesis project focuses on synthesizing a conceptual framework for evolvability and empirically investigating the relationship between developmental canalization against environmental perturbation and evolvability.
Developmental canalization against environmental perturbation has been hypothesized to promote the accumulation of neutral genetic variation in a population, a phenomenon that promotes evolvability.
I aim to asses the feasibility of leveraging this theoretical construct to promote evolvability in EANN, an interesting (and potentially useful) line of inquiry in the field of EANN.
This work also has broader relevance, particularly in the field of evolutionary biology, which is grappling with the relationship between environmental influence on the phenotype and evolution.
