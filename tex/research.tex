%\section{Research History}
My first foray into research was with the Fowler Laboratory at Oregon State University during the summers of 2011 and 2012.
My research goal was to identify proteins that interact with the exocyst complex.
We worked with populations of \textit{Arabidopsis thaliana}, a model organism in plant biology, that harbored a known mild mutation directly affecting the exocyst complex.
I screened these populations for a 13:3:1 phenotypic ratio, which would be telltale of synergistic interaction between the known exocyst mutation and a mildly deleterious mutation at an independently-assorting locus.
This entailed examining the translucent roots of dozens of delicate seedlings on a parafilm-sealed petri dishes for characteristically stumpy mutants.
I relished the purpose and discipline of this work.
After performing a statistical analysis of my observations, I flagged two populations for further analysis.
By identifying new exocyst interactors, we aimed to shed light on the role of the exocyst in cellular biology.
Although not immediately translating to technological or commercial ends, as fundamental research such work is key to opening avenues of inquiry for future study and, ultimately, in paving the way for other more applied advances.
After my time at the Fowler lab, I continued working in botany and horticulture for two years, studying the molecular biology of vineyard grape development with the Deluc laboratory and developing berry cultivars for the fresh fruit and processed food markets with the USDA Finn group.
From my work in laboratory biology, I gained expertise in experimental design and methodology.
I have found that this expertise sets me apart in the domains of computer science and mathematics.
I aim to leverage this edge throughout my graduate research career.

Fast forward to my REU with the Swarm Lab at NJIT.
I sat at my new desk in the open-plan office after a preliminary meeting with visiting mathematics professor Jason Graham and primary investigator Simon Garnier consumed by a swirl of surprise, adrenaline, and, also, a little panic.
The meeting had boiled down to this:
``Here's the problem.
Here are some resources.
Take a crack at it.''
Although I was initially surprised and intimidated by the extent of the autonomy and discretion they entrusted to me, in retrospect I wouldn't have had it any other way;
ownership of my work was personally rewarding and, I believe, ultimately contributed to its quality.
My task was to extend mathematical models of ant foraging, which are well-developed on flat surfaces, to uneven terrains.
It is well known that, as a collective, ants can optimize the foraging path they travel between nest and food.
On flat terrain, a clear ``best'' path exists: the shortest-distance foraging path, the most energy-efficient path, and the quickest-trip path are all identical.
On uneven terrains, however, this is no longer necessarily the case.
For example, an obstacle may make a trip along the most direct path take longer than a trip that circumvents it.
In the absence of a singular ``best'' path, the question of which trade-offs ants make --- and how they make them --- is of great interest.
After surveying existing models of ant foraging behavior on flat terrain and individual ant behavior on inclined surfaces, I designed and numerically evaluated a differential equations-based model of the foraging behavior of ants over uneven terrain.
Analysis of time series of ant positions and orientations generated via simulation revealed that as the severity of the incline the ants traverse increases, the model predicts that ants will tend to favor a more direct and less variable foraging path.

My model will see continued use, enabling the Swarm Lab to understand how real ants make collective foraging decisions on uneven terrain by simulating hypothesized behaviors and assessing the resulting foraging path predictions in comparison with upcoming \textit{in vivo} foraging experiments.
Ultimately, research into the collective intelligence of insects translates directly to technological applications;
for example, such research has been leveraged in swarm robotics projects, such as NASA's ant-inspired ``Swarmies'' that may one day harvest resources for Martian colonies and distributed traffic management schemes (especially in relation to autonomous vehicles).
On a personal level, I strengthened my self-confidence and self-management.
Importantly, I also learned when and how to ask for help.
I felt empowered by the process of translating fragmentary ideas into a vision and then into results.
During my time at MSU, I am looking forward to continuing to learn to operate this idea-vision-results pipeline and, especially, having the free reign to run it full bore.
