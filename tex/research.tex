%\section{Research History}
My first foray into research was with the Fowler Laboratory at Oregon State University during the summers of 2011 and 2012.
I recall, while examining the translucent roots of a few dozen delicate seedlings on parafilm-sealed petri dishes for characteristically stumpy mutants, experiencing a sense of exhilarating self-awareness; I relished the purpose, discipline, and curiosity of my work.
My research goal was to identify proteins that interacts interact with the exocyst complex.
I screened populations of \textit{Arabidopsis thaliana}, a model organism in plant biology, for a 13:3:1 phenotypic ratio telltale of synergistic interaction between a known mutation to the exocyst complex and a mildly deleterious mutation at an independently-assorting locus.
After performing a statistical analysis of my observations, I flagged two populations for further analysis.
My mentor, research assistant Rex Cole, and I hoped that identification of novel exocyst interactors, or confirmation of known exocyst interactors, would shed light on the role of the exocyst in plants.
Although not immediately translating to technological or commercial ends, as fundamental research such work is key to opening avenues of inquiry for future study and, ultimately, in paving the way for other more applied advances.
After my time at the Fowler lab, I continued working in botany and horticulture for two years, studying the molecular biology of vineyard grape development with the Deluc laboratory and developing berry cultivars for the fresh fruit and processed food markets with the USDA Finn group.
From my work in laboratory biology, I became versed in the carefully-considered design and methodological meticulousness experimental inquiry requires, which, I believe, gives me an edge in the domains of computer science and mathematics and which I look forward to continuing to put to use in graduate school.

Fast forward to my REU with the Swarm Lab at NJIT last summer.
I sat at my new desk in the open-plan office after a preliminary meeting with visiting mathematics professor Jason Graham and primary investigator Simon Garnier consumed by a swirl of surprise, adrenaline, and, also, a little panic.
The meeting boiled down to this:
``Here's the problem.
Here are some resources.
Take a crack at it.''
Although I was initially surprised and intimidated by the extent of the autonomy and discretion they entrusted to me, in retrospect I wouldn't have had it any other way; owning my work was personally rewarding and, I believe, ultimately contributed to its quality.
My task was to extend mathematical models of ant foraging, which are well-developed on flat surfaces, to uneven terrains.
It is well known that, as a collective, ants can optimize the foraging path they travel between nest and food.
On flat terrain, a clear ``best'' path exists: the shortest-distance foraging path, the most energy-efficient path, and the quickest-trip path are all identical.
On uneven terrains, however, this is no longer necessarily the case and the question of which trade-offs ants make --- and how they make them --- is of great interest.
After surveying existing models of ant foraging behavior on flat terrain and individual ant behavior on inclined surfaces, I designed and numerically evaluated a differential equations-based model of the foraging behavior of ants over uneven terrain.
Analysis of time series of ant positions and orientations generated via simulation revealed that as the severity of the incline the ants traverse increases, the model predicts that ants will tend to favor a more direct and less variable foraging path.

My model will see continued use, enabling the Swarm Lab to work out the rules by which real ants act on uneven terrain in a foraging context by simulating hypothesized behaviors and assessing the resulting foraging path predictions in comparison with upcoming in vivo ant foraging experiments.
Ultimately, research into the collective intelligence of insects translates directly to technological applications; for example, such research has been leveraged in swarm robotics projects, such as NASA's ant-inspired ``Swarmies'' that may one day harvest resources for Martian colonies and distributed traffic management schemes (especially in relation to autonomous vehicles).
On a personal level, this research experience has translated into a penchant for autonomously chewing on difficult problems, which will serve me well throughout graduate school.
