Smoke might as well have been coming out of my ears.
When I'm thinking, really thinking, long pauses between clauses throw a jagged edge into my cadence.
I'll tap my forehead, squint up and turn towards the ceiling, even work through a sentence in several drafts before announcing that, yes, that's what I want to say.
I had come in to meet with my undergraduate thesis advisor, Professor America Chambers, ready to discuss several papers on the Hyper-NEAT encoding scheme and derivative techniques.
We were up at the white board in her office, walking through an illustrative example employing the Hyper-NEAT scheme.
She sat down, put her chin in her hand, and piercingly scrutinized the board.
``So, why would you want to do this? What does it get you?" she asked.
``Well,'' I started.
``Well.
Hmmm.''
I turned to the board and rolled the Expo marker in my palm.
Then, speaking more quickly,
``What it really does... it biases the search space.
It makes regular phenotypic outcomes more easily accessible, more likely to occur...
So for problems where good solutions exhibit phenotypic regularity, those solutions become easier to find.''
We continued back and forth, keeping on for a few more minutes.

As I was packing up, Dr. Chambers grinned and asked ``When you're with your friends, do you think so hard about your words?''
I took a few seconds to consider.
``No, I don't think so.
Not as much, anyways.''
It wrings me out, but this is what I love.
I love topics that trip me up and conversations that make me grope for my words.
I especially love plugging into the challenging, collaborative intellectual environment cultivated in the scientific community to distill diffuse curiosity into direct, actionable questions.

My current research interest is seeking out what's missing from simple descriptions of evolution built on bare bones selection, variation, and inheritance.
Working with advisor Dr. America Chambers, I synthesized a conceptual framework to understand evolvability as my undergraduate thesis work.
Evolvability, the relationship between the configuration of a system and the phenotypic outcomes of mutational perturbation to that system, has been connected to an overwhelmingly large array of causal factors.
I organized this constellation of causal factors into three larger classes and described broader relationships between these classes.
My subsequent capstone project empirically investigated the relationship between environmental influence on the phenotype and evolvability.
In experiments performed with a gene regulatory network model, I found that populations evolved with stochastic perturbations to the development process had a higher incidence of silent mutation.
I hypothesize that modeling disruptive environmental influence on the phenotype introduced selection on internal structural characteristics --- specifically, against unstable cyclic regulatory interactions --- that otherwise would have remained invisible.
These internal structural characteristics, in turn, affect the phenotypic outcomes observed under mutation, in this case increasing the frequency at which mutations had no phenotypic effect.
Working with the same model, I also found that populations evolved under a selective pressure to respond to environmental signals by activating alternate developmental pathways were more sensitive to mutation.

My capstone and thesis work contribute to active conversations about evolvability taking place among and between evolutionary computing and evolutionary biology researchers.
Evolvability has immediate practical implications to evolutionary computing.
Classic techniques, such as genetic programming, are largely built on the bare bones understanding of evolution.
Translating more nuanced elements of the evolutionary process like environmental influence on the phenotype, by increasing the novelty and usefulness of mutational outcomes, will yield more powerful digital evolution techniques.
Evolutionary biology, in turn, is grappling with a number of recent, controversial extensions to the canonical modern synthesis.
Key to evaluating these extensions are questions of the evolutionary origins of evolvability and the role of environmental influence on the phenotype in the evolution of novel traits.

I use computational models in my research because they offer a unique ability to put a finger on what's missing in terms of explanatory power.
I honed my skill applying computational techniques to biological questions with the Swarm lab at NJIT.
I remember my first meeting with advisors Simon Garnier and Jason Graham.
The meeting had boiled down to this:
``Here's the problem.
Here are some resources.
Take a crack at it.''
Although I was initially surprised and intimidated by the extent of the autonomy and discretion they entrusted to me, in retrospect I wouldn't have had it any other way;
ownership of my work was personally rewarding and, I believe, ultimately contributed to its quality.
My task was to extend mathematical models of ant foraging, which are well-developed on flat surfaces, to uneven terrains.
It is well known that, as a collective, ants can optimize the foraging path they travel between nest and food.
On flat terrain, a clear ``best'' path exists: the shortest-distance foraging path, the most energy-efficient path, and the quickest-trip path are all identical.
On uneven terrains, however, this is no longer necessarily the case.
For example, an obstacle may make a trip along the most direct path take longer than a trip that circumvents it.
In the absence of a singular ``best'' path, the question of which trade-offs ants make --- and how they make them --- is of great interest.
After surveying existing models of ant foraging behavior on flat terrain and individual ant behavior on inclined surfaces, I designed and numerically evaluated a differential equations-based model of the foraging behavior of ants over uneven terrain.
Analysis of time series of ant positions and orientations generated via simulation revealed that as the severity of the incline the ants traverse increases, the model predicts that ants will tend to favor a more direct and less variable foraging path.

My model will see continued use, enabling the Swarm Lab to understand how real ants make collective foraging decisions on uneven terrain by simulating hypothesized behaviors and assessing the resulting foraging path predictions in comparison with upcoming \textit{in vivo} foraging experiments.
Ultimately, research into the collective intelligence of insects translates directly to technological applications;
for example, such research has been leveraged in swarm robotics projects, such as NASA's ant-inspired ``Swarmies'' that may one day harvest resources for Martian colonies and distributed traffic management schemes (especially in relation to autonomous vehicles).
On a personal level, I felt empowered by the process of translating fragmentary ideas into a vision and then into results.
During my graduate career, I am looking forward to continuing to learn to operate this idea-vision-results pipeline and, especially, having the free reign to run it full bore.

I first honed my skills designing and performing experiments I used in my research on ant foraging and evolvability through work in botany with the Fowler laboratory at Oregon State.
My research goal was to identify proteins that interact with the exocyst complex.
We worked with populations of \textit{Arabidopsis thaliana}, a model organism in plant biology, that harbored a known mild mutation directly affecting the exocyst complex.
I screened these populations for a 13:3:1 phenotypic ratio, which would be telltale of synergistic interaction between the known exocyst mutation and a mildly deleterious mutation at an independently-assorting locus.
This entailed examining the translucent roots of dozens of delicate seedlings on a parafilm-sealed petri dishes for characteristically stumpy mutants.
I relished the purpose and discipline of this work.
After performing a statistical analysis of my observations, I flagged two populations for further analysis.
Studying one of these populations, the group isolated a novel Golgi-localized protein that was confirmed to synergisticaly interact with the exocyst.
Although not immediately translating to technological or commercial ends, as fundamental research such work is key to opening avenues of inquiry for future study and, ultimately, in paving the way for other more applied advances.
Our work has helped nail down scientific understanding of the cellular systems the exocyst participates in and opens a host of the previously uncharacterized protein.
In addition to expertise in experimental design and methodology, from my time with the Fowler laboratory, I took away a sense and role model for community.

When I joined up with the Fowler group, I felt right at home.
In sixth grade, when I started to develop an interest in biology, I also began playing the oboe.
Band kids root in an intense culture built on cooperative competition,
These values are perpetuated among scientists and musicians alike by tradition and necessity.
How else can we develop extensive technical skills, push ourselves to do our best work, and come together to build something larger than ourselves?
I love the willingness to invest, high expectations, and mutual respect scientists and musicians alike hold for their kind.
It was these values, which I found in the Fowler group and the other teams I have worked with since, that inspired me to turn my interest in science into a career in science.
Independent of passion for my research interests, I cherish my membership in the scientific community in and of itself.
To me, belong as a scientist acting is completely entwined with acting to further these values and to bring new members into our fold.

Over the course of my undergraduate career, I built relationships with younger students at the University of Puget Sound and the Tacoma School District as a subject tutor, an academic consultant, an AP tutor, a classroom assistant, a musical coach, a Q\&A panelist, and an after-school club leader.
For me, these relationships became part of what it meant to be a student.
Today, I see them as the most important accomplishments of my undergraduate career.

In particular, I remember being caught off guard by a student I was talking to about triangles at Oakland, an alternative high school.
Out of the blue, she asked, ``What's so great about college?''
In that moment, I almost launched into a biographical account of why \textit{I} liked it.
Instead, I asked, ``What's your favorite thing to talk about?''
She took a moment, then replied, ''Books, I guess.''
She elaborated on her interest for stories and characters, elaborating on a few of her favorite pieces of young adult fiction, and bragged a little about reaching the cap for books that could be checked out of the library at once.
``What's so cool,'' I told her, ``is that you can take classes in whatever you want.
You can take classes about stories and hang out with other people who love them too.''
We went back and forth about \textit{really} being able to study what you want a few times before she seemed somewhat satisfied.
Before returning to triangles, I suggested that she would really get along with my humanities friends studying literature,
``That's really what they like to talk about.
All the time.
You'd fit in.''
She helped me realize that sharing a personal connection, a feeling, or enthusiasm is often more important than dispensing advice and information.
I began to recognize similar dynamics with the STEM students I tutored.
At the end of an some of my appointments, our conversation would shift to upper division coursework, techniques, and concepts that lay years ahead for them.
I took these conversations as opportunities to establish us as members of the same intellectual community instead of opportunities to dispense graduation requirements and course descriptions.
Although these were small gestures, I feel proud to have to put out the metaphorical welcome mat.

As scientists, we also have the opportunity to  intellectual impact of our work can be made tangible through technology.
Although my interest lies in fundamental research, by probing the algorithmic principles of biology I hope to contribute to the development of more capable and versatile artificial intelligence systems.
Beneath my own curiosity and ambition, I feel a moral imperative to participate in this work.
Motivations of scientists --- especially those working in technology-driven fields --- can often come off as cerebral and abstract.
(I recall, in particular, a talk that closed with a back-of-a-napkin calculation of relativistic constraints on the maximum number of human lives that could be created over lifespan of the universe).
I strive to check this tendency in myself.

When reflecting on the potential human impact of advances in computing, I am reminded of my father, who volunteers with the Dial-A-Bus program in Benton county.
He serves individuals who would be otherwise unable to get around town, curtailing their ability to fully participate in the community.
I hope that, through applications that counteract disability and free us from dangerous or simply menial tasks, stronger artificial intelligence will enable people to more fully exercise their human capabilities and, thereby, lay a fuller claim to their humanity.
I see glimmers of progress towards this goal in the work I've done with the Swarm Lab, the Fowler lab, and, now, the Ofria lab.

To translate my research into applied advances in artificial intelligence, I will work in close collaboration with industry.
I participated in such work during my my time with the Finn small fruits breeding laboratory at the USDA.
Pushing our berry cultivars out into the fresh fruit and processed food markets requires extensive and deliberate legwork.
We hosted regular symposiums around brief and actionable presentations on best practices and made a point to always wrap up with taste-tests berry sauce with ice cream.
We made long trips to growers' homesteads to exchange biological samples and chitchat.
We secured patent protections for our new cultivars to support a more entrepreneurial funding model for our group.
We spent time brainstorming catchy names for new cultivars that memorably allude to their best traits.
I learned how to develop personal relationships, how to concretely explain --- and demonstrate --- a compelling use case for my work in the context of \textit{their practices}.
The scale our agricultural partners can achieve made our work at the USDA even more rewarding;
we get to see our cultivars fill vast fields that sweep over the horizon and, ultimately, in muffins, ice cream, farmers' markets, and roadside stands.
From my vantage, I see a wave of bio-inspired computing techniques sweeping through industry.
I am ready to work with industrial partners to add my research to that wave.

My affiliation with the NSF BEACON Center for the study of evolution in action provides a strong foundation for the goals I have laid out for my graduate career.
My field --- particularly the BEACON community --- is rife with opportunities for collaborations across disciplines and institutions.
I am particularly interested in collaborating with biologists to perform wet work to better understand what properties of the genotype-phenotype mapping facilitate evolution and working with to hardware/electrical engineers to explore distributed system design for swarm computing.
GRFP funding would provide a strong position to develop and take advantage of relationships with science and industry collaborators.
In graduate school, I want to facilitate similar experiences for other students through near-peer mentoring relationships.
BEACON provides opportunities, particularly summer research funding for dozens of undergraduates from underrepresented backgrounds, to develop these relationships.
I will invest time and resources made available by GRFP support into these relationships.
I feel strongly that education, outreach, mentorship, and community are intrinsic to what it means do science.
At a scale larger than myself, I believe the scientific community stands to benefit by affording the same intrinsic value we place in the traditional, intellectual aspects of science to these activities and even more deeply incorporating them into our work.
GRFP support help me lead in the conversation about the intrinsic value of outreach, education, mentorship, and community in science.
I see my graduate career as an unparalleled opportunity to --- harnessing a more nuanced understanding of evolution to a more fleshed-out scientific appreciation for evolution in nature and practical problem-solving computational techniques and strengthening the intellectual community doing this work.
The latitude GRFP support provides, in conjunction with dedicated resources available through the NSF BEACON, will kick-start my graduate ambitions.
