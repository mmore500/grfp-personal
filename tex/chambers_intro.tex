I take jagged, long pauses between clauses when I'm thinking hard.
I'll tap my forehead, squint up and turn towards the ceiling, even work through a sentence in several drafts before announcing that, yes, that's what I want to say.
I had come in to meet with my undergraduate thesis advisor, Professor America Chambers, ready to discuss several papers on Hyper-NEAT.
This indirect genotype-to-phenotype encoding scheme was designed to bring the regularity and patterning observed in biology to artificial neuroevolution.
We were up at the white board in her office, walking through an illustrative example.
She sat down, put her chin in her hand, and piercingly scrutinized the board.
``So, why would you want to do this? What does it get you?'' she asked.
``Well,'' I started.
``Well.
Hmmm.''
I turned to the board and rolled the Expo marker in my palm.
It wrings me out, but this is what I love.
I love abstracting the astonishing structures and mechanisms embedded at every level of biological organization.
I love and conversations that  trip me up and make me grope for my words.
I love plugging into the challenging, collaborative intellectual environment cultivated in the scientific community to distill diffuse curiosity into direct, actionable questions.
As a scientist, I aspire
(1) to contribute to conversation about fundamental evolutionary mechanics and bio-inspired approaches to artificial intelligence through digial evolution research
and (2) to build an inclusive, intellectually vibrant scientific community through mentoring relationships and STEM education.
