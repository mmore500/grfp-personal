When I think hard, I take long pauses between clauses.
I tap my forehead, squint at the ceiling, even compose a sentence in several drafts aloud before deciding that, yes, that \textit{is} what I want to say.
I had come in to meet with my undergraduate thesis advisor, Professor America Chambers, ready to discuss several papers on Hyper-NEAT.
This indirect encoding scheme for artificial neuroevolution was designed to harness the regularity and patterning observed in biology.
We were up at the white board in her office, walking through an illustrative example.
She sat down, put her chin in her hand, and scrutinized the board.
``So, why would you want to do this? What does it get you?'' she asked.
``Well,'' I started.
``Well.
Hmmm.''
I turned to the board and rolled the Expo marker in my palm.
It wrings me out, but this is what I love.
I left that conversation with Dr. Chambers with my finger on how the Hyper-NEAT technique promotes fruitful neuroevolution, not just how it mimics biology.
I love and conversations that trip me up, make me grope for my words, and force me to crystallize my thoughts.
I love the challenging, collaborative intellectual environment cultivated in the scientific community to distill diffuse curiosity into actionable questions.
I love abstracting the astonishing structures and mechanisms embedded at every level of biological organization.
As a scientist, I aspire
(1) to develop fundamental theory and applications for bio-inspired approaches to artificial intelligence and
(2) to promote an inclusive and intellectually vibrant scientific community through mentoring relationships and STEM education.
