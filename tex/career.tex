%\section{Career Aspirations}
Although my interest lies in fundamental research, by porting the design principles of biology over to algorithm design I hope to contribute to the development of more capable and versatile artificial intelligence systems.
Beneath my own curiosity and ambition, I feel a moral imperative to participate in this work.
In my experience, the motivations of scientists --- especially those working in technology-driven fields --- can often come off as cerebral and abstract.
(I recall, in particular, attending a talk that closed with back-of-a-napkin calculation of the limit imposed by relativistic constraints on the maximum number of human lives we could create over the lifespan of the universe).
I strive to check this tendency in myself by exercising self-awareness and engaging directly with causes I believe in, such as youth outreach.
When reflecting on the human impact of artificial intelligence, I am reminded of my father, who volunteers with the Dial-A-Bus program in Benton county.
He serves individuals who would be otherwise unable to get around town, curtailing their ability to fully participate in the community.
I hope that, through applications such as autonomous vehicles that counteract disability and free us from dangerous or simply menial tasks, stronger artificial intelligence will enable people to more fully exercise their human capabilities and, thereby, lay a fuller claim to their humanity.

%\subsection{How NSF GRFP funding would help}

I am excited to continue exploring the field of evolutionary computation through graduate education, an experience which GRFP funding would turbocharge.
Beyond freeing up my time to focus on research, it would allow me to study at institutions with smaller computer science departments where funding might be less certain but which are forerunners in the field of evolutionary computing, such the University of Wyoming.
Not only would these programs align best with my interests, but they would provide the type of close-knit community that I thrive in.


My ultimate career aspiration is to channel my graduate education into a career in industry.
In just the last few years there has been a surge of bio-AI activity in the technology sector; large technology companies such as Google and Facebook have jumped into the field and new players that specialize in bio-AI, such as evolutionary computing-focused Sentient Technologies, have come to the table.
The results that corporate investment has already yielded, such as AlphaGo's besting of human play at the notoriously difficult game of Go or human-level image recognition, are already concrete, abundant, and --- moreover --- exciting.
The electrifying potential in industry stems from the unique outlet it provides to connect theory and application and from the unparalleled resources it musters --- in terms of computational power and data, but also in terms of talent;
industry hosts a thriving and collaborative community of thought-leaders in AI, a group which I plan to join.
In the meantime, I am excited to jump into studying bio-inspired approaches to AI at the graduate level.
NSF GRFP support would allow me to hit the ground running.
