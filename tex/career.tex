%\section{Career Aspirations}
Although my interest lies in fundamental research, by probing the algorithmic principles of biology I hope to contribute to their harnessing for the development of more capable and versatile artificial intelligence systems.
My ultimate career aspiration is to channel my graduate education into a work in, or in close collaboration with, industry.
Beneath my own curiosity and ambition, I feel a moral imperative to participate in this work.

When reflecting on the human impact of artificial intelligence, I am reminded of my father, who volunteers with the Dial-A-Bus program in Benton county.
He serves individuals who would be otherwise unable to get around town, curtailing their ability to fully participate in the community.
I hope that, through applications that counteract disability and free us from dangerous or simply menial tasks, stronger artificial intelligence will enable people to more fully exercise their human capabilities and, thereby, lay a fuller claim to their humanity.

In my experience, the motivations of scientists --- especially those working in technology-driven fields --- can often come off as cerebral and abstract.
(I recall, in particular, attending a talk that closed with a back-of-a-napkin calculation of the limit imposed by relativistic constraints on the maximum number of human lives we could create over the lifespan of the universe).
I strive to check this tendency in myself by engaging directly with causes I believe in, such as youth outreach.
Over the course of my undergraduate career, I have jumped at many opportunities to get to know and mentor Tacoma-area students: as an AP tutor, a classroom assistant, a musical coach, a Q\&A panelist, and an after-school club leader.
I feel very strongly about affording other students the same opportunities to exercise their capabilities in music, science, and math that I have enjoyed.
