%\section{Career Aspirations}
%\section{Career Aspirations}
Although my interest lies in fundamental research, by probing the algorithmic principles of biology I hope to enable the development of more capable and versatile artificial intelligence systems.
Beneath my own curiosity and ambition, I feel a moral imperative to participate in this work.
Motivations of scientists --- especially those working in technology-driven fields --- can often come off as cerebral and abstract.
(I recall, in particular, a talk that closed with a back-of-a-napkin calculation of relativistic constraints on the maximum number of human lives that could be created over lifespan of the universe).
I strive to check this tendency in myself.

When reflecting on the potential human impact of advances in computing, I am reminded of my father, who volunteers with the Dial-A-Bus program in Benton county.
He serves individuals who would be otherwise unable to get around town, curtailing their ability to fully participate in the community.
I hope that, through applications that counteract disability and free us from dangerous or simply menial tasks, stronger artificial intelligence will enable people to more fully exercise their human capabilities and, thereby, lay a fuller claim to their humanity.
I see glimmers of progress towards this goal in the work I've done with the Swarm Lab, the Fowler lab, and, now, the Ofria lab.

To translate my research into applied advances in artificial intelligence, I will work in close collaboration with industry.
I learned what working with industry means during my my time with the Finn small fruits breeding laboratory at the USDA.
It means making long trips to their homestead to exchange biological samples and chitchat.
It means walking growers through your field and hawking animated accounts of the up and coming crosses they should keep an eye out for.
It means organizing regular symposiums around brief and actionable presentations on best practices.
Of course, it means making a point to wrap up those symposiums by taste-testing berry sauce with ice cream.
It means continually brainstorming catchy names for new cultivars that memorably allude to their best traits.
It means tabulating thorn density on 10cm lengths of cane in service of patent applications for those cultivars
I know how to develop personal relationships then concretely explain --- and demonstrate --- a compelling use case for my work in the context of \textit{their practices} then make it easy --- and exciting --- to incorporate my work into their practices.
Collaborating with industry also means acres and acres of new berry cultivars plantings that over the horizon; it means new berry cultivars in muffins, ice cream, farmers markets, and roadside stands.
I am eager to continue to exploit collaboration with industry during my graduate career and beyond.
The latitude GRFP support provides, in conjunction with dedicated resources available through the NSF BEACON center for the study of evolution in action , will kick-start my efforts.
