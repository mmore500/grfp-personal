\textbf{Impact}
When I joined up with the Fowler group, I felt right at home.
In sixth grade, when I started to develop an interest in biology, I also began playing the oboe.
Band kids root in an intense culture built on mentoring relationships, and friendly warmth between peers.
These values are perpetuated among scientists and musicians alike by tradition and necessity.
How else can we develop extensive technical skills, push ourselves to do our best work, and come together to build something larger than ourselves?
I love the willingness to invest, high expectations, and mutual respect scientists and musicians alike hold for their kind.
It was these values, which I found in the Fowler group and the other teams I have worked with since, that inspired me to turn my interest in science into a career in science.
Independent of passion for my research interests, I cherish my membership in the scientific community in and of itself.
To me, belong as a scientist acting is completely entwined with acting to further these values and to bring new members into our fold.

Over the course of my undergraduate career, I built relationships with younger students at the University of Puget Sound and the Tacoma School District as a subject tutor, an academic consultant, an AP tutor, a classroom assistant, a musical coach, a Q\&A panelist, and an after-school club leader.
For me, these relationships became part of what it meant to be a student.
Today, I see them as the most important accomplishments of my undergraduate career.

In particular, I remember being caught off guard by a student I was talking to about triangles at Oakland, an alternative high school.
Out of the blue, she asked, ``What's so great about college?''
In that moment, I almost launched into a biographical account of why \textit{I} liked it.
Instead, I asked, ``What's your favorite thing to talk about?''
She took a moment, then replied, ''Books, I guess.''
She elaborated on her interest for stories and characters, elaborating on a few of her favorite pieces of young adult fiction, and bragged a little about reaching the cap for books that could be checked out of the library at once.
``What's so cool,'' I told her, ``is that you can take classes in whatever you want.
You can take classes about stories and hang out with other people who love them too.''
We went back and forth about \textit{really} being able to study what you want a few times before she seemed somewhat satisfied.
Before returning to triangles, I suggested that she would really get along with my humanities friends studying literature,
``That's really what they like to talk about.
All the time.
You'd fit in.''
She helped me realize that sharing a personal connection, a feeling, or enthusiasm is often more important than dispensing advice and information.
I began to recognize similar dynamics with the STEM students I tutored.
At the end of an some of my appointments, our conversation would shift to upper division coursework, techniques, and concepts that lay years ahead for them.
I took these conversations as opportunities to establish us as members of the same intellectual community instead of opportunities to dispense graduation requirements and course descriptions.
Although these were small gestures, I feel proud to have to put out the metaphorical welcome mat.

I am committed to continuing my work fostering community among scientists and performing STEM outreach.
I currently volunteer teacher's assistant with special education and general education mathematics classrooms in the East Lansing school district.
As I progress in my graduate studies, I am especially looking forward to mentoring undergraduate researchers as well as welcoming and supporting new BEACON graduate students.
Further into the future, I see taking on a leadership role in a research group as an opportunity to establish the type of community I have enjoyed belonging to.
As I make a foothold in my own career, I strive to follow in the footsteps of my role models and reach out to bring new members into the communities that I care about.
I am concerned about barriers to inclusivity in the scientific community.
Lived experience as a LGBT person has deepened my perspective on these barriers.
I have come to realize that identity is not baggage, to be checked at the door but must be recognized as an inherent part of belonging to the scientific community.
Although the barriers I have encountered pale in comparison to those faced by others, I hope to leverage these molehills to reach out over mountains.
