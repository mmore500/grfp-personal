\textbf{Impact}
As an undergraduate, I built relationships with high school and college students as a subject tutor, an academic consultant, an AP tutor, a classroom assistant, a musical coach, a Q\&A panelist, and an after-school club leader.
For me, these relationships are part of what it meant to be a student.
Today, I see them among the most important accomplishments of my undergraduate career.

In particular, I remember being caught off guard by a student I was talking to about triangles at Oakland, an alternative high school.
Out of the blue, she asked, ``What's so great about college?''
In that moment, I almost launched into a biographical account of why \textit{I} liked it.
Instead, I asked, ``What's your favorite thing to talk about?''
She took a moment, then replied, ''Books, I guess.''
She elaborated on her interest for stories and characters, listing a few of her favorite pieces of young adult fiction, and bragged about reaching the cap for books that could be checked out of the library at once.
``What's so cool,'' I told her, ``is that you can take classes in whatever you want.
You can take classes about stories with other people who love them too.''
We went back and forth about \textit{really} being able to study what you want a few times before she seemed somewhat satisfied.
Before returning to triangles, I suggested that she would really get along with my humanities friends studying literature,
``That's really what they talk about.
All the time.
You'd fit in.''
She helped me realize that sharing a personal connection, a feeling, or enthusiasm is often more important than dispensing advice.
I began to recognize similar dynamics with the STEM students I tutored.
At the end of an some of my appointments, our conversation would shift to upper division coursework  and concepts that lay years ahead for them.
I took these conversations as opportunities to establish us as members of the same intellectual community instead of opportunities to dispense graduation requirements and course descriptions.
Although these were small gestures, I feel proud to have to put out the metaphorical welcome mat.

I am committed to continuing my work fostering community among scientists and performing STEM outreach.
I currently volunteer as a teacher's assistant with special education and general education classrooms in the East Lansing school district.
As I progress in my graduate studies, I look forward to mentoring undergraduate researchers and new graduate students.
In the future, I see leading a research group as an opportunity to establish the type of community I have enjoy belonging to.
I strive to follow in the footsteps of my role models and reach out to bring new members into the scientific community.
I am concerned about barriers to inclusivity in the scientific community.
My experience as a LGBT person has deepened my perspective on these barriers.
I have come to realize that identity is not baggage, to be checked at the door but must be recognized as an inherent part of belonging to the scientific community.
Although the barriers I have encountered pale in comparison to those faced by others, I hope to leverage these molehills to reach out over mountains.
