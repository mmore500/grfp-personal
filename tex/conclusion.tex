%\subsection{How NSF GRFP funding would help}
Although intellectual impact can be made tangible through technology, as scientists we must also recognize the immediate impacts we can make through choices about how we choose to conduct ourselves and our research activities.
Over the course of my undergraduate career, I built relationships with younger students at the University of Puget Sound and the Tacoma School District as a subject tutor, an academic consultant, an AP tutor, a classroom assistant, a musical coach, a Q\&A panelist, and an after-school club leader.
For me, these relationships became part of what it meant to be a student.
Today, I see them as the most important accomplishments of my undergraduate career.

In particular, I remember being caught off guard by a student I was talking to about triangles at Oakland, an alternative high school.
Out of the blue, she asked, ``What's so great about college?''
In that moment, I almost launched into a biographical account of why \textit{I} liked it.
Instead, I asked, ``What's your favorite thing to talk about?''
She took a moment, then replied, ''Books, I guess.''
She elaborated on her interest for stories and characters, elaborating on a few of her favorite pieces of young adult fiction, and bragged a little about reaching the cap for books that could be checked out of the library at once.
``What's so cool,'' I told her, ``is that you can take classes in whatever you want.
You can take classes about stories and hang out with other people who love them too.''
We went back and forth about \textit{really} being able to study what you want a few times before she seemed somewhat satisfied.
Before returning to triangles, I suggested that she would really get along with my humanities friends studying literature,
``That's really what they like to talk about.
All the time.
You'd fit in.''
She helped me realize that sharing a personal connection, a feeling, or enthusiasm is often more important than dispensing advice and information.
I began to recognize similar dynamics with the STEM students I tutored.
At the end of an some of my appointments, our conversation would shift to upper division coursework, techniques, and concepts that lay years ahead for them.
I took these conversations as opportunities to establish us as members of the same intellectual community instead of opportunities to dispense graduation requirements and course descriptions.
Although these were small gestures, I feel proud to have to put out the metaphorical welcome mat.

In graduate school, I want to facilitate similar experiences for other students through near-peer mentoring relationships.
The NSF BEACON Center provides opportunities, particularly summer research funding for dozens of undergraduates from underrepresented backgrounds, to develop these relationships.
I will invest time and resources made available by GRFP support into these relationships.
I feel strongly that education, outreach, and mentorship are intrinsic to what it means do science.
At a scale larger than myself, I believe the scientific community stands to benefit by affording the same intrinsic value we place in the traditional, intellectual aspects of science to these activities and even more deeply incorporating them into our work.
For example, I began drawing freehand cartoons illustrating my work for presentations and articles aimed at a broader audience.
To my surprise, I found they became important tools to engage a scientific audience, as well.
They have even become useful to my own understanding of topics I'm working on.
I am particularly inspired by the leadership of the NSF BEACON center in this direction by, for example, repackaging research software as educational tools and leading outdoor field trips for schoolkids to collect data.
GRFP support will help me lead by example to grow the conversation about what it means to do science.

I also care about taking ownership of my idea-vision-results pipeline.
My field --- particularly the BEACON community --- is rife with opportunities for collaborations across disciplines and institutions.
I am particularly interested in collaborating with biologists to perform wet work to better understand what properties of the genotype-phenotype mapping facilitate evolution and working with to hardware/electrical engineers to explore distributed system design for swarm computing.
GRFP funding would provide a strong position to develop and exploit these relationships.
As I begin my graduate career, I am grateful for the opportunity to join in exploring the intersection between computing and evolution and join up with the community doing this work.
GRFP funding will support my ambitions for my graduate career: to lead in strengthening the scientific community and to lead in organizing intellectual collaboration.
