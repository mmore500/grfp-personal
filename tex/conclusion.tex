%\subsection{How NSF GRFP funding would help}
Although intellectual impact can be made tangible through technology, as scientists we must, but often fail to, recognize important dimensions to personal achievement beyond intellectual contribution.
That narrative restricts what it means to do science to activity that directly generates intellectual results, casting the work that nourishes the community that houses those ideas and carries them forward --- outreach, education, and mentorship --- in competition for time and resources with ``real'' science.
True, our community is already strengthened by work done under the auspices of investment for future intellectual return, public relations, checking boxes to receive grant funding, and the warmhearted generosity of many scientists.
However, we can accomplish education, outreach, and mentorship more pervasively and more effectively if they are afforded the same intrinsic value we place in science.
To strengthen our scientific community, the matrix in which intellectual science exists, we must recognize outreach, education, and mentorship as an intrinsic part of doing science.

Over the course of my undergraduate career, I built relationships with younger students at the University of Puget Sound and the Tacoma School District as a subject tutor, an academic consultant, an AP tutor, a classroom assistant, a musical coach, a Q\&A panelist, and an after-school club leader.
I view these relationships as the most important accomplishments of my undergraduate career.
In particular, as a subject tutor I was caught off guard by the handful of students who, usually at the end of an appointment, asked about navigating upper division coursework and material.
Even though these topics were remote --- years away for these students --- I consistently felt that something more profound than a rote exchange of graduation requirements and course descriptions had transpired.
I only recently put a finger on it.
At the core, these conversations established us as members of the same intellectual community.
Although a small gesture, I feel proud to have to put out the metaphorical   welcome mat.

It is difficult to overstate the importance of belonging.
In graduate school, I want to facilitate similar experiences for other students through near-peer mentoring relationships.
The NSF BEACON Center for the study of evolution in action provides opportunities, particularly summer research funding for dozens of undergraduates from underrepresented backgrounds, to develop these relationships.
I will invest time and resources made available by GRFP support into these relationships.
In the conversation about what it means to do science, I will lead by example.

I also care deeply about taking ownership of my idea-vision-results pipeline.
My field --- particularly the BEACON community --- is rife with opportunities for collaborations across disciplines and institutions.
I am particularly interested in collaborating with biologists to perform wet work to better understand what properties of the genotype-phenotype mapping facilitate evolution and working with to hardware/electrical engineers to explore distributed system design for swarm computing.
GRFP funding would provide a strong position to develop and exploit these relationships.
As I begin my graduate career, I am grateful for the opportunity to join in exploring the intersection between computing and evolution and join up with the community doing this work.
GRFP funding will support my ambitions for my graduate career: to lead in strengthening the scientific community and to lead in organizing intellectual collaboration.
