\textbf{Conclusion}
My affiliation with BEACON provides a strong foundation for the goals I have laid out for my graduate career.
My field --- particularly the BEACON community --- is rife with opportunities for collaborations across disciplines and institutions.
I am particularly interested in collaborating with biologists to perform wet work to better understand what properties of the genotype-phenotype mapping facilitate evolution and working with to hardware/electrical engineers to explore distributed system design for swarm computing.
GRFP funding would provide a strong position to develop and take advantage of relationships with science and industry collaborators.
In graduate school, I want to facilitate similar experiences for other students through near-peer mentoring relationships.
BEACON provides opportunities, particularly summer research funding for dozens of undergraduates from underrepresented backgrounds, to develop these relationships.
I will invest time and resources made available by GRFP support into these relationships.
I feel strongly that education, outreach, mentorship, and community are intrinsic to what it means do science.
At a scale larger than myself, I believe the scientific community stands to benefit by affording the same intrinsic value we place in the traditional, intellectual aspects of science to these activities and even more deeply incorporating them into our work.
GRFP support help me lead in the conversation about the intrinsic value of outreach, education, mentorship, and community in science.
I see my graduate career as an unparalleled opportunity to --- harnessing a more nuanced understanding of evolution to a more fleshed-out scientific appreciation for evolution in nature and practical problem-solving computational techniques and strengthening the intellectual community doing this work.
The latitude GRFP support provides, in conjunction with dedicated resources available through the NSF BEACON, will kick-start my graduate ambitions.
