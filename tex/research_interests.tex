\textbf{Research Interests}
I am interested in understanding what critical nuances are missing from simple algorithmic implementations of evolution built on bare bones selection, variation, and inheritance.
As an undergraduate advised by Dr. America Chambers, I synthesized a conceptual framework to understand evolvability, the potential for adaptive change to occur in an organism's descendants.
The origins of evolvability have been connected to a large array of causal factors.
I organized this constellation of causal factors into three larger classes and analyzed their broad relationships.
For my subsequent capstone project, I used a gene-regulatory network model to empirically investigate the relationship between environmental influence on the phenotype and evolvability.
I found that populations subjected to stochastic perturbations to the development process evolved a higher incidence of silent mutation.
I hypothesized that disruptive environmental influence on the phenotype selected against unstable cyclic regulatory interactions that otherwise would have remained invisible, thus increasing the frequency at which mutations had no phenotypic effect.
Working with the same model, I also found that organisms evolved to respond to environmental signals by activating alternate developmental pathways were more sensitive to mutation.

My research contributes to ongoing scientific conversations about evolvability.
Evolutionary biologists see evolvability as central to understanding the evolution of complex traits such as human intelligence.
Questions about the role of environmental influence on the phenotype in the evolution of novel traits, in particular, are central to debate over recent, controversial extensions to the canonical modern evolutionary synthesis.
In turn, evolutionary computing researchers seek more evolvable systems to solve practical problems.
Translating more nuanced elements of the evolutionary process like environmental influence on the phenotype will yield powerful applied evolution techniques.
I am a doctoral student at Michigan State University, working with Dr. Charles Ofria in the NSF BEACON Center for the study of evolution in action.
I am uniquely positioned to contribute to conversations about evolutionary dynamics to apply novel biological concepts to evolutionary computation.
