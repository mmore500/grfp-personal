\textbf{Research Interests}
My current research interest is seeking out what's missing from simple descriptions of evolution built on bare bones selection, variation, and inheritance.
Working with advisor Dr. America Chambers, I synthesized a conceptual framework to understand evolvability as my undergraduate thesis work.
Evolvability, the relationship between the configuration of a system and the phenotypic outcomes of mutational perturbation to that system, has been connected to an overwhelmingly large array of causal factors.
I organized this constellation of causal factors into three larger classes and described broader relationships between these classes.
My subsequent capstone project empirically investigated the relationship between environmental influence on the phenotype and evolvability.
In experiments performed with a gene regulatory network model, I found that populations evolved with stochastic perturbations to the development process had a higher incidence of silent mutation.
I hypothesize that modeling disruptive environmental influence on the phenotype introduced selection on internal structural characteristics --- specifically, against unstable cyclic regulatory interactions --- that otherwise would have remained invisible.
These internal structural characteristics, in turn, affect the phenotypic outcomes observed under mutation, in this case increasing the frequency at which mutations had no phenotypic effect.
Working with the same model, I also found that populations evolved under a selective pressure to respond to environmental signals by activating alternate developmental pathways were more sensitive to mutation.

My capstone and thesis work contribute to active conversations about evolvability taking place among and between evolutionary computing and evolutionary biology researchers.
Evolvability has immediate practical implications to evolutionary computing.
Classic techniques, such as genetic programming, are largely built on the bare bones understanding of evolution.
Translating more nuanced elements of the evolutionary process like environmental influence on the phenotype, by increasing the novelty and usefulness of mutational outcomes, will yield more powerful digital evolution techniques.
Evolutionary biology, in turn, is grappling with a number of recent, controversial extensions to the canonical modern synthesis.
Key to evaluating these extensions are questions of the evolutionary origins of evolvability and the role of environmental influence on the phenotype in the evolution of novel traits.
I am currently a doctoral student at Michigan State University, working with Dr. Charles Ofria's digital evolution laboratory.
Our group is closely affiliated with the NSF BEACON Center for the study of evolution in action.
I am uniquely positioned to contribute to conversations about evolutionary mechanics among biologists and digital evolution researchers.
