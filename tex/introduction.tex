In sixth grade, I was up late to hot glue tiny beads from my family's junk drawer onto the cardboard cell model that I had traced out.
While I struggled to manipulate and place the small plastic pieces into my model, I also struggled to cobble together the rather simplistic descriptions of organelles we had learned into an understanding of the cellular whole.
It felt like building an intellectual house of cards.
Because every piece of the cell depends on others, I felt as if I could not truly understand any piece in isolation.
I repeatedly failed to make a purchase.
I felt ready to abandon this bizarre, confusing puzzle and leave it far out of mind.
In the next breath, though, I began thinking down a different track.
Evidence that complex, stochastic systems work permeates our living world.
It hit me: answers to the question ``how?'' must be everywhere, too, if you just look.
From that glimmer of curiosity onward, the astonishing structures and mechanisms embedded at every level of biological organization have continuously delighted me.
In particular, I love thinking about and experimenting with abstractions of the astonishing structures and mechanisms embedded at every level of biological organization.
Computer science provides the vehicle for this endeavor.

When I started to develop an interest in biology, I also began playing the oboe.
Like science, band culture is built on mentoring relationships, cooperative competition, and friendly warmth between peers.
These values are perpetuated among scientists and musicians alike by tradition and necessity.
How else can we develop extensive technical skills, push ourselves to do our best work, and accomplish goals beyond the capabilities of any single individual?
I love the willingness to invest, high expectations, and mutual respect scientists and musicians alike hold for their kind.
It was these pervasive values that inspired me to turn my interest in science into a career.

As a scientist, I aspire
(1) to develop fundamental theory and applications for bio-inspired approaches to artificial intelligence and
(2) to promote an inclusive and intellectually vibrant scientific community through mentoring relationships and STEM education.
