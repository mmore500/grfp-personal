%\section{Introduction/Self-Reflection}

In sixth grade, I was up late to hot glue tiny beads from my family's junk drawer onto the cardboard cell model that I had traced out.
While I struggled to manipulate and place the small plastic pieces into my model, I also struggled to cobble together the rather simplistic descriptions of organelles we had learned into an understanding of the cellular whole.
It felt like building an intellectual house of cards.
Because every piece of the cell depends on others, I felt as if I could not truly understand any piece in isolation.
I repeatedly failed to make a purchase.
I felt ready to abandon this bizarre, confusing puzzle and leave it far out of mind.
In the next breath, though, I began thinking down a different track.
Evidence that complex, stochastic systems work permeates our living world.
It hit me: answers to the question ``how?'' must be everywhere, too, if you just look.
From that glimmer of curiosity onward, the astonishing structures and mechanisms embedded at every level of biological organization have continuously delighted me.
I have found studying biology to be exploration in the truest, most adventurous, sense of the word.

In sixth grade, when I got my start in biology, I also began playing the oboe.
Band kids root in an intense culture built on cooperative competition, mentoring relationships, and friendly warmth between peers.
Several years later, when I joined up with a botany research group, I felt right at home.
These values are perpetuated among scientists and musicians alike by tradition and necessity.
How else can we develop extensive technical skills, push ourselves to do our best work, and come together to build something larger than ourselves?
I love the willingness to invest, high expectations, and mutual respect scientists and musicians alike hold for their kind.
It was these values that inspired me to turn my interest in science into a career in science.
Independent of passion for my research interests, I cherish my membership in the scientific community in and of itself.
To me, belong as a scientist acting is completely entwined with acting to further these values and to bring new members into our fold.
