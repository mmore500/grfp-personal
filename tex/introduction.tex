In sixth grade, I was up late to glue tiny beads from my family's junk drawer onto the cardboard cell model that I had traced out.
While I struggled place the small plastic pieces onto my model, I also struggled to cobble together the simplistic descriptions of organelles we had learned into an understanding of the cellular whole.
Because every piece of the cell depends on others, I felt I could not truly understand any piece in isolation.
That experience stoked a glimmer of curiosity that grew into a love of thinking about and experimenting with abstractions of the \textbf{astonishing structures and mechanisms embedded at every level of biological organization}.
Today, \textbf{computational experiments} are my vehicle for this endeavor.

When I started to develop an interest in biology, I also began playing the oboe.
Like science, band culture is built on \textbf{mentoring relationships, cooperative competition, and friendly warmth between peers}.
Without this community, we could not develop extensive technical skills, push ourselves to do our best work, and accomplish goals far beyond the capabilities of any single individual.
The willingness to invest, high expectations, and mutual respect scientists hold for their kind inspired me to pursue a career in science.

As a scientist, I aspire
(1) to \textbf{develop fundamental theory and applications for bio-inspired approaches to AI} and
(2) to \textbf{promote an inclusive, intellectually vibrant science community through mentoring relationships and STEM education}.
