%\section{Introduction/Self-Reflection}

In sixth grade, I was up late to hot glue tiny beads from my family's junk drawer onto the cardboard cell model I had traced out.
While I struggled to manipulate and place the small pieces, I also struggled to piece together an intuition for how cellular pieces constituted a cellular whole from the rather simplistic descriptions of organelles we had learned.
It felt like an intellectual exercise in arch building.
I repeatedly failed to make a purchase;
because every piece of the cell depends on others, no piece could be truly understood in isolation.
My frustration turned to resigned disbelief.
In the next breath, though, I began thinking down a different track.
Evidence that systems governed by stochastic interactions between innumerable independent components can support cohesive and dynamic functionality permeates our living world.
It hit me: answers to the question ``how?'' do too.
From that point onward, the astonishing structures and mechanisms embedded at every level of biological organization have continuously delighted me.
I have found studying biology to be exploration in the truest, most adventurous, sense of the word.

In sixth grade, when I got my start in biology, I also began playing the oboe.
Band kids root in an intense culture built on cooperative competition, mentoring relationships, and friendly warmth between peers.
Several years later, when I got the opportunity to join the Fowler lab in the Botany department at Oregon State University as a research assistant, I felt right at home.
These values are perpetuated among scientists and band members by tradition and necessity.
How else can we develop extensive technical skills, push ourselves to do our best work, and come together to build something larger than ourselves?
I love the willingness to invest, high expectations, and mutual respect scientists and musicians alike hold for their kind.
For these reasons, I cherish my membership in these communities.
The most rewarding aspect of belonging, I have found, is the responsibility of upholding these values.
My responsibility to others, not just how I am treated by others, has become fundamental to my understanding of belonging itself.
I believe it is these responsibilities that define and bind the scientific community and, importantly, paves the way for the inclusion of new members.

I also gravitated to music because it brings together the acts of exploring and creating.
Studying the theory and history of a piece --- exploring --- and rehearsing and performing it --- creating --- fundamentally inform one another.
As an undergraduate, my scientific interest shifted to straddle boundaries between evolutionary biology and computing.
This position offers a unique opportunity to bring together exploring and creating.
Computational models of evolution inform our understanding of biology and biology inspires computational tools built on evolutionary principles.
My career aspiration is in the field bio-AI, harnessing biologically-inspired approaches to create intelligent systems.
I believe that my interdisciplinary interests and my commitment to community set me apart and will continue to distinguish my work.
