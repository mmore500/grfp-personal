%\section{Introduction/Self-Reflection}

My scientific interest and work straddle disciplinary boundaries between biology and computing and, more broadly, between engineering and science.
I love to both explore and to create.
From many years of playing the oboe, I would describe science and engineering like studying the theory and history of a piece and performing it; exploring and creating inform each other and, together, are electric.
In sixth grade, right while I began playing the oboe, I also got my start in biology.
I remember hot gluing a collection of tiny beads to my cardboard cell model, struggling to deduce how the rather simplistic descriptions of organelles we had learned could account for the entirety of a cohesive system.
My first reaction was disbelief that a system governed by random interactions between innumerable independent components could function cohesively.
In the next breath, though, I realized it must be possible; evidence --- and, ultimately, answers to the question of ``how?'' --- permeates our living world.
From that point onward, the astonishing structures and mechanisms embedded at every level of biological organization have continuously delighted me;
in particular, I am captivated by the paradigm of emergence that underlies biological systems.
I find studying and researching biology to be exploration in the truest, most adventurous, sense of the word.
My career aspiration is in the field bio-AI, harnessing biologically-inspired approaches for the design of intelligent systems.
I believe that my unique set of interests, straddling between biology and computing and, more broadly, between engineering and science, set me apart and will continue to distinguish my work.
