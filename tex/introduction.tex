%\section{Introduction/Self-Reflection}

My scientific interest straddles disciplinary boundaries between biology and computing and, more broadly, between engineering and science.
I gravitate to the intersection of exploration and creation.
I see a parallel intersection in music.
Science is like studying the theory and history of a piece.
Engineering is like rehearsing and performing it.
Exploring and creating inform each other and, together, are electric.
In sixth grade, when I began playing the oboe, I got my start in biology.
I remember staying up late to hot glue tiny beads from my family's junk drawer cardboard cell model I had traced out.
All the while, I struggled to piece together WHAT from the rather simplistic descriptions of organelles we had learned.
My initial reaction was disbelief.
In the next breath, though, I began thinking down a different track.
Evidence that systems governed by random interactions between innumerable independent components can support cohesive and dynamic functionality permeates our living world.
It hit me: answers to the question ``how?'' do too.
From that point onward, the astonishing structures and mechanisms embedded at every level of biological organization have continuously delighted me.
I have found studying biology to be exploration in the truest, most adventurous, sense of the word.
My career aspiration is in the field bio-AI, harnessing biologically-inspired approaches to create intelligent systems.
I believe that my unique set of interests, straddling between biology and computing and, more broadly, between engineering and science, set me apart and will continue to distinguish my work.
