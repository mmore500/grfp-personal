\textbf{Career Goals}
My current research interests directly stem from my undergraduate thesis and capstone work.
I am pursing a doctoral degree with advisor Dr. Charles Ofria in the NSF BEACON Center for the study of evolution in action at Michigan State University.
The BEACON Center's high performance computing resources and interdisciplinary community uniquely position me collaborate to collaborate with biologists on \textit{in vivo} work and with evolutionary computing practitioners to apply novel biological concepts to evolutionary computation.
I am interested in understanding what critical nuances are missing from simple algorithmic implementations of evolution built on bare bones selection, variation, and inheritance;
my specific focus is on environmental influence in the genotype-phenotype mapping.

By probing the algorithmic principles of biology I hope to contribute to the development of more capable and versatile artificial intelligence systems.
Beneath my own curiosity and ambition, I feel a moral imperative to participate in this work.
When reflecting on the potential human impact of advances in computing, I am reminded of my father, who volunteers with the Dial-A-Bus program in Benton county.
He serves individuals who would be otherwise unable to get around town, curtailing their ability to fully participate in the community.
I hope that, through applications that counteract disability and free us from dangerous or simply menial tasks, stronger artificial intelligence will enable people to more fully exercise their human capabilities and, thereby, lay a fuller claim to their humanity.
I see glimmers of progress towards this goal in the work I've done with the USDA, the Swarm Lab, and, now, the Ofria lab.
I will use the skills I gained through the USDA and the Mathematical Competition in Modeling to apply my research to real-world artificial intelligence problems.

I am committed to continuing my work fostering community among scientists and performing STEM outreach.
I currently volunteer four hours a week as a teacher's assistant with special education and general education classrooms in the East Lansing school district.
As I progress in my graduate studies, I look forward to mentoring undergraduate researchers and new graduate students.
In the future, I see leading a research group as an opportunity to establish the type of community I have enjoy belonging to.
I strive to follow in the footsteps of my role models and reach out to bring new members into the scientific community.
I am concerned about barriers to inclusivity in the scientific community.
My experience as a LGBT person has deepened my perspective on these barriers.
I have come to realize that identity is not baggage, to be checked at the door but must be recognized as an inherent part of belonging to the scientific community.
Although the barriers I have encountered pale in comparison to those faced by others, I hope to leverage these molehills to reach out over mountains.

Long term, I see continuing in academic research as a primary investigator as the best route to accomplish my outreach, technological, and scientific goals.
