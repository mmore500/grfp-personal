\noindent
\textit{\textbf{Career Goals}}
My current research interests directly stem from my undergraduate thesis and capstone work.
\textbf{I am pursing a doctoral degree with advisor Dr. Charles Ofria in the NSF BEACON Center for the study of evolution in action at Michigan State University}.
The BEACON Center's high performance computing resources and interdisciplinary community uniquely position me to collaborate with biologists on \textit{in vivo} work and with evolutionary computing practitioners to apply novel biological concepts to evolutionary computation.
I am interested in understanding what critical nuances are missing from simple algorithmic implementations of evolution built on bare bones selection, variation, and inheritance;
my specific focus is on \textbf{evolvability and environmental influence in the genotype-phenotype mapping}.
Understanding these phenomena will help biologists answer questions about the evolution of complex traits like human intelligence and will help evolutionary computing researchers develop more powerful applied evolution techniques to solve practical problems.

By probing the algorithmic principles of biology, I hope to contribute to the \textbf{development of more capable, versatile AI systems} with direct human impact.
I am reminded of my father, who volunteers for Dial-A-Bus to serve individuals whose disabilities would otherwise limit their ability to fully participate in the community.
I hope that, through applications that counteract disability and free us from dangerous or simply menial tasks, stronger AI will enable people to more fully exercise their human capabilities.
My work with the Swarm lab, Dr. Chambers, and, now, the Ofria lab helps lay the groundwork for this future.
I will leverage my outreach skills build industrial collaborations that apply my research to real-world AI problems.

I am committed to \textbf{continuing my work fostering community among scientists and performing STEM outreach}.
I currently volunteer four hours a week as a teacher's assistant in special and general education classrooms in East Lansing.
In my third period classroom, my work is specially targeted at engaging minority students.
As I progress in my graduate studies, I look forward to welcoming new graduate students and taking advantage of BEACON funding support to mentor undergraduates on summer research projects.

Ultimately, I aspire to lead an academic research group and build an inclusive community that will allow me and others to accomplish our outreach, technological, and scientific goals.
GRFP support will help me \textbf{conduct cutting edge research at the intersection of biology and computer science} an lead in the conversation about the intrinsic value of \textbf{outreach, education, mentorship, and community in science}.
