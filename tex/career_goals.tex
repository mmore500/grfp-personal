\textit{\textbf{Career Goals}}
My current research interests directly stem from my undergraduate thesis and capstone work.
\textbf{I am pursing a doctoral degree with advisor Dr. Charles Ofria in the NSF BEACON Center for the study of evolution in action at Michigan State University}.
The BEACON Center's high performance computing resources and interdisciplinary community uniquely position me collaborate to collaborate with biologists on \textit{in vivo} work and with evolutionary computing practitioners to apply novel biological concepts to evolutionary computation.
I am interested in understanding what critical nuances are missing from simple algorithmic implementations of evolution built on bare bones selection, variation, and inheritance;
my specific focus is on \textbf{evolvability and environmental influence in the genotype-phenotype mapping}.
Understanding these phenomena will help biologists answer questions about the evolution of complex traits like human intelligence and will help evolutionary computing researchers develop more powerful applied evolution techniques to solve practical problems.

By probing the algorithmic principles of biology I hope to contribute to the \textbf{development of more capable and versatile AI systems}.
When reflecting on the potential human impact of advances in computing, I am reminded of my father, who volunteers with the Dial-A-Bus program in Benton county.
He serves individuals who would be otherwise unable to get around town, curtailing their ability to fully participate in the community.
I hope that, through applications that counteract disability and free us from dangerous or simply menial tasks, stronger AI will enable people to more fully exercise their human capabilities.
My work with the Swarm lab, Dr. Chambers, and, now, the Ofria lab helps lay the groundwork for this future.
I will leverage my experience with the USDA and MCM to build industrial collaborations that apply my research to real-world AI problems.

I am committed to \textbf{continuing my work fostering community among scientists and performing STEM outreach}.
I currently volunteer four hours a week as a teacher's assistant in special and general education classrooms in East Lansing.
As I progress in my graduate studies, I look forward to welcoming new graduate students and taking advantage of BEACON's funding support to mentor undergraduates on summer research projects.
I am concerned about barriers to inclusivity in the scientific community.
My experience as a LGBT person has deepened my perspective on these barriers.
I believe identity is not baggage to be checked at the door but must be recognized and respected.
Although the barriers I have encountered pale in comparison to those faced by others, I hope to leverage these molehills to reach out over mountains.
