\textbf{Research History}
I study computational models because they offer experimental control, transparency, and speed resulting in the potential for substantial explanatory power.
I honed my biological modeling skills working with the Swarm lab at NJIT.
I remember my first meeting with advisors Simon Garnier and Jason Graham.
They told me,
``Here's the problem.
Here are some resources.
Take a crack at it.''
I was initially surprised and intimidated by the extent of the autonomy they entrusted to me, but in retrospect owning my work was personally rewarding and ultimately contributed to its quality.
I extended mathematical models of ant foraging to consider uneven terrains.
It is well known that, as a collective, ants can optimize the foraging path they travel between nest and food.
On flat terrain, the shortest-distance foraging path, the most energy-efficient path, and the quickest path are all identical.
On uneven terrains, however, an obstacle may make the most direct path take longer than a trip that circumvents it.
In the absence of an absolute ``best'' path, the question of how ants make trade-offs is of great interest.
After surveying existing models of ant foraging behavior, I designed and numerically evaluated a differential equations-based model of the foraging behavior of ants over uneven terrain.
Analysis of time series of ant positions and orientations generated via my simulation predict on severe inclines, ants will tend to favor a more direct and less variable foraging path. My model will next be used to compare collective foraging decisions of simulated ants on uneven terrain with upcoming \textit{in vivo} foraging experiments.

Research into the collective intelligence of insects translates directly to technological applications;
as examples, NASA's ant-inspired robot ``Swarmies'' may one day harvest resources for Martian colonies and Honda is currently developing ``Safe swarm'' technology to reduce traffic delays and hazards through direct vehicle-to-vehicle cooperation.
I felt empowered to be part of the process of translating fragmentary ideas into a vision and then into results.
I look forward to continuing to learn to operate this idea-vision-results pipeline and, especially, having the free reign to run it full bore.

I first honed my skills designing and performing experiments through work in botany with the Fowler laboratory at Oregon State.
My research goal was .
We worked with populations of \textit{Arabidopsis thaliana}, a model organism in plant biology, to identify proteins that interact with the exocyst complex.
I screened populations that harbored a known mild mutation directly affecting the exocyst for a 13:3:1 phenotypic ratio, which would be telltale of synergistic interaction between the known exocyst mutation and a mildly deleterious mutation at an independently-assorting locus.
This entailed examining the translucent roots of dozens of delicate seedlings on a parafilm-sealed petri dishes for characteristically stumpy mutants.
I relished the purpose and discipline of this work.
I performed statistical analysis of my observations, flagging two populations for further analysis.
Studying one of these populations, the group isolated a novel Golgi-localized protein that was confirmed to synergisticaly interact with the exocyst.
Our work has helped nail down scientific understanding of the cellular systems the exocyst participates in and opens a host of the previously uncharacterized protein.
Although not immediately translating to technological or commercial ends, as fundamental research such work is key to opening avenues of inquiry for future study and, ultimately, in paving the way for other more applied advances.
In addition to expertise in experimental design and methodology, from my time with the Fowler laboratory I took away a strong impression of what a friendly, supportive, and inclusive scientific community looks like.
