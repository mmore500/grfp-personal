\textbf{Research History}
I began my research career investigating the exocyst complex in plants with the Fowler laboratory at Oregon State University through a local Apprenticeships in Science program.
I used genetic assays and phenotypic measurements to screen 25 populations of \textit{A thaliana} for synergistic interaction between a known exocyst mutation and a mildly deleterious mutation at an independently-assorting locus.
The group went on to isolate a novel Golgi-localized protein confirmed to interact with the exocyst \cite{fowler}.
From my time with the Fowler laboratory I took away a strong impression of what a friendly, supportive, and inclusive scientific community looks like.
The experience has also become a very personal exhibit of the impact of STEM outreach.
I feel strongly about making similar opportunities available to others.

Beginning my college career, I worked two full-time summers with the USDA small fruits breeding laboratory.
The group, led by Dr. Chad Finn, develops berry cultivars for the fresh fruit and processed food markets.
Realizing widespread production of those cultivars requires deliberate cooperation with partner growers.
Among other efforts, we hosted regular symposiums around brief and actionable presentations on best practices and made a point to always wrap up with taste-tests of berry sauce with ice cream.
Later in my undergraduate career, I exercised my experience at the interface of science and industry through three bouts in the Mathematical Competition in Modeling (MCM).
These four day sprints, completed in teams of three students, emphasize pitching insights from mathematical models to business executives and policy makers.
In 2017, we developed a model of traffic in the greater Seattle area and showed that in the near future designating lanes exclusively for autonomous vehicles will reduce commuter travel delays.
We received a Finalist award, ranking among the top 11 of 1,527 participating teams.
These experiences showed me the importance of developing personal relationships and purposefully demonstrating a compelling use case for my work in the context of the target audience's perspectives and practices.

In my Sophomore year, I worked with Dr. Smith at University of Puget Sound to design a project to develop methods for automated isolation of mouse
ultrasonic vocalizations (USVs) from noisy recordings.
I won NASA funding through a competitive grant application process.
Existing software tools to identify and characterize USVs, which are used as a quantitative assay for the affective and social state of mice in biomedical research, were readily confounded by background noise.
I developed and tested filtering algorithms inspired by the Sobel Edge detection method that, after being trained on human-annotated spectrograms of mouse vocalizations, distinguish between true mouse vocalization signals and background noise, achieving 75\% accuracy at 25\% recall.
The project culminated in a poster presentation on campus \cite{smith}.
I took away a set of concrete computational research skills, in particular learning data management, version control, and visualization techniques.

I brought together my biological and computational interests together at the NJIT Swarm lab.
I was recruited by advisors Dr. Garnier and Dr. Graham through a REU coordinated by the Mathematical Biosciences Institute.
I extended mathematical models of ant foraging to consider uneven terrains.
It is well known that, as a collective, ants can optimize the foraging path they travel between nest and food.
On flat terrain, the shortest-distance foraging path, the most energy-efficient path, and the quickest path are all identical.
On uneven terrains, however, an obstacle may make the most direct path take longer than a trip that circumvents it.
In the absence of an absolute ``best'' path, the question of how ants make trade-offs is of great interest.
After surveying existing models of ant foraging behavior, I designed and numerically evaluated a differential equations-based model of the foraging behavior of ants over uneven terrain.
My simulation predicts that on severe inclines ants will tend to favor a more direct and less variable foraging path.
My work culminated in presentations at an undergraduate symposium organized by the mathematical biosciences institute and at the 2017 Joint Mathematics Meetings \cite{mbi, jmm}.

My model will next be used to compare collective foraging decisions of simulated ants on uneven terrain with upcoming \textit{in vivo} foraging experiments.
Research into the collective intelligence of insects translates directly to technological applications;
as examples, NASA's ant-inspired robot ``Swarmies'' may one day harvest resources for Martian colonies and Honda is currently developing ``Safe swarm'' technology to reduce traffic delays and hazards through direct vehicle-to-vehicle cooperation.
Reflecting on my time with the Swarm lab, I was initially surprised and intimidated by the extent of the autonomy they entrusted to me, but in retrospect owning my work was personally rewarding and ultimately contributed to its quality.
I felt empowered to be part of the process of translating fragmentary ideas into a vision and then into results.
I look forward to continuing to learn to operate this idea-vision-results pipeline and, especially, having the free reign to run it full bore.

Working with advisor by Dr. Chambers on a senior thesis project, I synthesized a conceptual framework to understand evolvability, the potential for adaptive change to occur in an organism's descendants.
The origins of evolvability have been connected to a large array of causal factors.
I organized this constellation of causal factors into three larger classes and analyzed their broad relationships \cite{thesis}.
For my subsequent capstone project, I used a gene-regulatory network model to empirically investigate the relationship between environmental influence on the phenotype and evolvability.
I found that populations subjected to stochastic perturbations to the development process evolved a higher incidence of silent mutation.
I hypothesized that disruptive environmental influence on the phenotype selected against unstable cyclic regulatory interactions that otherwise would have remained invisible, thus increasing the frequency at which mutations had no phenotypic effect.
Working with the same model, I also found that organisms evolved to respond to environmental signals by activating alternate developmental pathways were more sensitive to mutation.
I presented these results at the 2017 BEACON Congress \cite{beacon}.
I found the union of experimental and theoretical work through my capstone and thesis projects extremely rewarding;
I want to keep doing theoretical work that informs my experimental work and vice versa.

This research contributes to ongoing scientific conversations about evolvability.
Evolutionary biologists see evolvability as central to understanding the evolution of complex traits such as human intelligence.
Questions about the role of environmental influence on the phenotype in the evolution of novel traits, in particular, are central to debate over recent, controversial extensions to the canonical modern evolutionary synthesis.
In turn, evolutionary computing researchers seek more evolvable systems to solve practical problems.
Translating more nuanced elements of the evolutionary process like environmental influence on the phenotype will yield powerful applied evolution techniques.
