\textit{\textbf{Research History}}
I began gaining research experience in high school.
Through a local \textbf{Apprenticeships in Science program}, I investigated the exocyst complex in plants with the Fowler lab at Oregon State University.
I used genetic assays and phenotypic measurements to screen 25 populations of \textit{Arabadopsis thaliana} for synergistic interaction between a known exocyst mutation and a mildly deleterious mutation at an independently-assorting locus.
The group went on to isolate a novel Golgi-localized protein confirmed to interact with the exocyst \cite{fowler}.
I took away a strong impression of what a friendly, supportive, and inclusive scientific community looks like.
This personal experience with STEM outreach motivates my desire to make similar opportunities available to others.

Beginning my college career, I worked two full-time summers with the \textbf{USDA small fruits breeding laboratory}.
The group, led by Dr. Chad Finn, develops berry cultivars for the fresh fruit and processed food markets.
Realizing widespread production of those cultivars requires deliberate cooperation with partner growers.
Among other efforts, we hosted regular symposiums around brief and actionable presentations on best practices and made a point to always wrap up with taste-tests of berry sauce with ice cream.
Later in my undergraduate career, I exercised my experience at the interface of science and industry through three bouts in the \textbf{Mathematical Competition in Modeling} (MCM).
These four day sprints, completed in teams of three students, emphasize pitching insights from mathematical models to business executives and policy makers.
In 2017, we developed a model of traffic in the greater Seattle area and showed that in the near future designating lanes exclusively for autonomous vehicles will reduce commuter travel delays.
We received a Finalist award, \textbf{ranking among the top 0.8\% of participating teams}.
These experiences showed me the importance of developing personal relationships and purposefully demonstrating a compelling use case for my work in the context of the target audience's perspectives and practices.%TODO

In my Sophomore year, I worked with Dr. Smith at University of Puget Sound (UPS) to \textbf{develop methods for automated isolation of mouse
ultrasonic vocalizations (USVs) from noisy recordings}.
I sought outside funding, and was \textbf{awarded NASA space grant funding of \$3,250} through a competitive application process.
Existing software tools identify and characterize USVs, but are often confounded by background noise.
I developed and tested filtering algorithms inspired by the Sobel Edge detection method that learn from human-annotated spectrograms distinguish between true mouse vocalization signals and background noise.
My approach achieved 75\% accuracy at 25\% recall from noisy recordings.
I presented these results at the UPS summer research symposium \cite{smith}.
I took away concrete computational research skills, including data management, version control, and visualization techniques.

I brought together my biological and computational interests together studying ant foraging behavior at the \textbf{NJIT Swarm lab}.
I was recruited by advisors Dr. Garnier and Dr. Graham through a REU funded by the Mathematical Biosciences Institute.
On flat terrain, the shortest-distance foraging path, the most energy-efficient path, and the quickest path are all identical.
However, on uneven terrains an obstacle may make the most direct path take longer than a trip that circumvents it.
In the absence of an absolute ``best'' path, the question of how ants make trade-offs is of great interest.
Thus, I extended computational models of ant foraging to consider uneven terrains.
My differential-equations based model predicts that severe inclines cause ants to favor a more direct, less variable foraging path.
My work culminated in presentations at the \textbf{2017 Joint Mathematics Meetings} \cite{jmm}.
During my time at the Swarm lab, I found the autonomy entrusted to me and the opportunity to answer unsolved biological questions that inform technologies like swarm robotics empowering and rewarding.
After this REU, I saw graduate school as the best opportunity to continue engaging in such self-directed, important work.

For my \textbf{senior thesis project} at UPS, I worked with Dr. Chambers to synthesize a conceptual framework to understand evolvability, the potential for adaptive change to occur in an organism's descendants.
Evolvability has been connected to a large array of causal factors.
I organized this constellation of causal factors into three larger classes and analyzed their broad relationships \cite{thesis}.
For my subsequent \textbf{capstone project}, I used a gene-regulatory network model to empirically investigate the how environmental influence on the phenotype and relates to evolvability.
I found that populations subjected to stochastic perturbations of the development process evolved a higher incidence of silent mutation and that populations evolved to respond to environmental signals by activating alternate developmental pathways were more sensitive to mutation.
I presented my capstone and thesis work at the 2016 NW Honors Symposium in Seattle, two campus seminars, and the 2017 BEACON Congress \cite{beacon}.
For my thesis and capstone work, I received the \textbf{MacArthur Award for an Outstanding Thesis Presentation} and the \textbf{Goman Outstanding Math/CS Senior Award}.
This fall, I have published my thesis work as an illustrated blog series aimed at both the general public and other scientists.
I found uniting experiments and theory in my thesis and capstone projects extremely rewarding;
this work inspired me to continue conducting research that tightly couples these domains.
