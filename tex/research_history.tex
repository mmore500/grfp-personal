\noindent
\underline{\smash{\textit{\textbf{Research History}}}}
I began gaining research experience as a high school student through an \textbf{Apprenticeships in Science program} at Oregon State University.
I worked with Dr. John Folwer to investigate the exocyst protein complex in plants.
Using genetic assays and phenotypic measurements, I screened 25 populations of \textit{Arabadopsis thaliana} for synergistic interactions between a known exocyst mutation and a mildly deleterious mutation at an independently-assorting locus.
The group went on to isolate a novel Golgi-localized protein confirmed to interact with the exocyst \cite{fowler}.
I took away a strong impression of what a friendly, supportive, and inclusive scientific community looks like.
This personal experience with STEM outreach motivates my desire to make similar opportunities available to others.

As an undergraduate, I worked at the interface of science and industry at the \textbf{USDA small fruits breeding laboratory}.
I spent two summers working full time with Dr. Chad Finn to develop berry cultivars for the fresh fruit and processed food markets.
Achieving widespread production of our cultivars requires buy-in from our partner growers.
To build these relationships, we hosted regular symposiums with brief and actionable presentations on best practices and actively involved growers in our scientific experiments.
I gained further experience making connections between the real word and science through three bouts in the \textbf{Mathematical Competition in Modeling} (MCM).
These four-day sprints emphasize pitching insights gained by developing and analyzing mathematical models to business executives and policy makers.
In 2017, my three-person team developed a model of highway traffic in the greater Seattle area and showed that in the near future designating lanes exclusively for autonomous vehicles will reduce commuter travel delays.
We \textbf{ranked among the top 11 of over 1,500 participating teams}.
I will continue to use my experience  building collaborative partnerships and gearing science toward a professional audience to translate my research into real-world innovations.

As a sophomore, I built my computational skill set \textbf{developing methods for automated isolation of mouse
ultrasonic vocalizations (USVs) from noisy recordings} with Dr. Adam Smith at University of Puget Sound (UPS).
I sought outside funding and was \textbf{awarded a NASA space grant of \$3,250} through a competitive application process.
USVs are an important quantitative assay for the affective and social state of mice in biomedical research, but existing software tools were readily confounded by background noise.
I developed and tested filtering algorithms inspired by the Sobel Edge detection method that use human-annotated spectrograms to learn to distinguish between true mouse vocalization signals and background noise.
My approach achieved 75\% accuracy at 25\% recall from noisy recordings.
I presented these results at the 2015 UPS summer research symposium \cite{smith}.
From this experience, I gained computational research skills including data management, version control, and visualization techniques.

I brought my biological and computational interests together studying ant foraging behavior at the \textbf{NJIT Swarm lab} in the summer after my junior year.
I was recruited by Drs. Simon Garnier and Jason Graham through the Mathematical Biosciences Institute REU.
For ants on flat terrain, the shortest-distance foraging path, the most energy-efficient foraging path, and the quickest foraging path are all identical.
However, on uneven terrain an obstacle may make the most direct path take longer than a trip that circumvents it.
In the absence of an absolute ``best'' path, the question of how ants make trade-offs is of great interest to biologists and engineers studying swarm robotics.
Thus, I extended computational models of ant foraging to consider uneven terrains.
My differential-equations based model predicts that severe inclines cause ants to favor a more direct, less variable foraging path.
I presented my work at the \textbf{2017 Joint Mathematics Meetings} \cite{jmm}.
During my time at the Swarm lab, I was empowered by the autonomy entrusted to me and found the opportunity to answer open biological questions with practical applications rewarding.
After this REU, I saw graduate school as the best way to continue engaging in such self-directed, impactful work.

For my \textbf{senior thesis project} at UPS, I worked with Dr. America Chambers to synthesize a conceptual framework for evolvability, the potential for adaptive change to occur in an organism's descendants.
Evolvability has been connected to a wide array of causal factors ranging from gene duplication to phenotypic plasticity.
I organized this constellation of causal factors into three larger classes and analyzed their broad relationships \cite{thesis}.
Building off my thesis, I conducted a \textbf{capstone project} to empirically investigate how environmental influence on the phenotype relates to evolvability in a gene-regulatory network model.
I found that populations subjected to stochastic perturbations during the development process evolved a higher incidence of silent mutation.
Further, populations pressured to respond to environmental signals by activating alternate developmental pathways were more sensitive to mutation.
I presented my findings at the 2016 NW Honors Symposium, two campus seminars, and the 2017 NSF BEACON Congress \cite{beacon}.
For my thesis and capstone work, I received the \textbf{MacArthur Award for an Outstanding Thesis Presentation} and the \textbf{Goman Outstanding Math/CS Senior Award}.
This fall, I have published my thesis work as an illustrated blog series aimed at both the general public and other scientists.
Uniting experiments and theory in my thesis and capstone projects was extremely rewarding;
this work inspired me to continue conducting research that tightly couples these domains.
